{
\setbeamertemplate{background} 
{
    \includegraphics[width=\paperwidth,height=\paperheight]{img/database-issues.jpg}
}
\begin{frame}
\end{frame}
}

\begin{frame}{La situaci\'on}
    Se desea desarrollar una base de datos para registrar las notas
    de los estudiantes de la carrera en cada una de las asignaturas que
    cursan:
    \begin{itemize}
        \item De cada estudiante se conoce su identificador, su nombre, su grupo y su provincia de residencia.
        \item De cada asignatura se conoce su identificador y su nombre.
        \item Por cada asignatura se conoce la nota que obtuvo el estudiante en la evaluaci\'on final.
    \end{itemize}
    Adem\'as, se conoce que los estudiantes son organizados en los grupos de acuerdo a su provincia.
\end{frame}


\begin{frame}{Primero lo primero}

    \resizebox{\linewidth}{!}{
                \begin{tikzpicture}[node distance=6em]
                    \tikzstyle{every entity} = [minimum width=2cm, minimum height=1.2cm]
                    \node[entity] (estudiante) {ESTUDIANTE}
                        [sibling distance=3cm]
                        child {node[attribute] [above right of=estudiante] {\tiny PROVINCIA}}
                        child {node[attribute] [above of=estudiante] {\tiny GRUPO}}
                        child {node[attribute] [above left of=estudiante] {\tiny ENOMBRE}}
                        child {node[attribute] [left of=estudiante] {\underline{\tiny \#E}}};
                  
                    \node[entity] (asignatura) at (8,0) {ASIGNATURA}
                    [sibling distance=3cm]
                    child {node[attribute] [right of=asignatura] {\underline{\tiny \#A}}}
                    child {node[attribute] [above right of=asignatura] {\tiny ANOMBRE}};
                    
                  
                    \node[relationship,aspect=2] (evaluar) at (4,0) {EVALUAR}[node distance=4em]
                    child {node[attribute] [above of=evaluar] {\tiny NOTA}};
                    \draw (evaluar.east) -- (asignatura.west) node[above left] {$0,\ast$};
                    \draw (evaluar.west) -- (estudiante.east) node[above right] {$0,\ast$};
                \end{tikzpicture}
            }
\end{frame}


\begin{frame}{Metodolog\'ia para obtener un esquema relacional correcto}
    \begin{enumerate}
        \item Identificar el universo $U$ de atributos del fen\'omeno.
        \item Identificar el conjunto $F$ de las dependencias funcionales que se establecen entre los atributos.
        \item Definir el esquema relacional $R(U,F)$.
    \end{enumerate}
\end{frame}


\begin{frame}{Ejemplo}
    \begin{enumerate}
        \item $U = \{ \textnormal{\#E, ENombre,  Grupo, Provincia, \#A, ANombre, Nota}\}$
    \end{enumerate}

    \onslide<2>{
    \vspace{10mm}

    \centering    
    \large{¿C\'omo podemos obtener $F$ a partir del dise\~no conceptual?}}
\end{frame}

\begin{frame}{Algoritmo de extracci\'on de dependencias funcionales}
    \begin{enumerate}
        \item<2-> Por cada conjunto de entidades con un conjunto de atributos $X \subseteq U$,
        se a\~nade la dependencia funcional $K \to X$ donde $K$ es la llave del conjunto de entidades.
        \item<3-> Por cada conjunto de interrelaciones se toma su llave $K$ y se a\~nade la
        dependencia funcional $K \to K$. Adem\'as, por cada conjunto de entidades en un extremo
        de cardinalidad m\'axima 1 en la interrelaci\'on, se a\~nade la dependencia funcional $K - K_E \to K_E$ donde
        $K_E$ es la llave del conjunto de entidades.
        \item<4-> Por cada agregaci\'on con un conjunto de atributos $X \subseteq U$ se a\~nade la dependencia
        funcional $K \to X$ donde $K$ es la llave del conjunto de interrelaciones que encierra la agregaci\'on.
        \item<5-> A\~nadir aquellas dependencias funcionales asociadas a otras restricciones del negocio
        especificadas en los requerimientos.
    \end{enumerate}
        
\end{frame}


\begin{frame}{Ejecutando el algoritmo}
    \centering
    \resizebox{!}{2.5cm}{
        \begin{tikzpicture}[node distance=6em]
            \tikzstyle{every entity} = [minimum width=2cm, minimum height=1.2cm]
            \node[entity] (estudiante) {ESTUDIANTE}
                [sibling distance=3cm]
                child {node[attribute] [above right of=estudiante] {\tiny PROVINCIA}}
                child {node[attribute] [above of=estudiante] {\tiny GRUPO}}
                child {node[attribute] [above left of=estudiante] {\tiny ENOMBRE}}
                child {node[attribute] [left of=estudiante] {\underline{\tiny \#E}}};
            
            \node[entity] (asignatura) at (8,0) {ASIGNATURA}
            [sibling distance=3cm]
            child {node[attribute] [right of=asignatura] {\underline{\tiny \#A}}}
            child {node[attribute] [above right of=asignatura] {\tiny ANOMBRE}};
            
            
            \node[relationship,aspect=2] (evaluar) at (4,0) {EVALUAR}[node distance=4em]
            child {node[attribute] [above of=evaluar] {\tiny NOTA}};
            \draw (evaluar.east) -- (asignatura.west) node[above left] {$0,\ast$};
            \draw (evaluar.west) -- (estudiante.east) node[above right] {$0,\ast$};
        \end{tikzpicture}
    }
        \vspace{3mm}


        \begin{enumerate}
            \item<2-> Se tienen los conjuntos de entidades ESTUDIANTE y ASIGNATURA:\begin{itemize}
                \item $\textnormal{\#E} \to \textnormal{ENombre, Grupo, Provincia}$
                \item $\textnormal{\#A} \to \textnormal{ANombre}$
            \end{itemize}
            \item<3-> Se tiene el conjunto de interrelaciones EVALUAR: \begin{itemize}
                \item $\textnormal{\#E,\#A} \to \textnormal{\#E,\#A}$
            \end{itemize}
            \item<4-> Se tiene la agregaci\'on ASIGNATURA-EVALUADA \begin{itemize}
                \item $\textnormal{\#E, \#A} \to \textnormal{Nota}$
            \end{itemize}
            \item<5-> A\~nadimos las restricciones planteadas en la especificaci\'on: \begin{itemize}
                \item $\textnormal{Provincia} \to \textnormal{Grupo}$
            \end{itemize}
        \end{enumerate}
\end{frame}

\begin{frame}{Continuemos con el ejemplo}
    \begin{enumerate}
        \item $U = \{ \textnormal{\#E, ENombre,  Grupo, Provincia, \#A, ANombre, Nota}\}$
        \item $F = \{$ \\
        \hspace{10mm} $\textnormal{\#E} \to \textnormal{ENombre, Grupo, Provincia}$\\
        \hspace{10mm} $\textnormal{\#A} \to \textnormal{ANombre}$\\
        \hspace{10mm} $\textnormal{\#E,\#A} \to \textnormal{\#E,\#A}$\\
        \hspace{10mm} $\textnormal{\#E, \#A} \to \textnormal{Nota}$\\
        \hspace{10mm} $\textnormal{Provincia} \to \textnormal{Grupo}$\\
        $\}$
        \item Definimos el esquema relacional $\textbf{Evaluaciones}(U,F)$ con llave \#E, \#A 
    \end{enumerate}

    \note{@NOTE aki' se pone una trivial expli'citamente (spoiler). Provincia->Grupo es una externa al merx}
\end{frame}

\begin{frame}{¿Es este un buen dise\~no? (Anomal\'ia de inserci\'on)}
    \centering
    \begin{tabular}{ccccccc}
        \underline{\#E} & ENombre & Grupo & Provincia & \underline{\#A} & ANombre & Nota\\
        $e_1$ & Juan & {111} & { La Habana} & $a_1$ & An\'alisis & 3\\
        $e_1$ & Juan & {111} & { La Habana} & $a_2$ & L\'ogica & 2\\
        $e_1$ & Juan & {111} & { La Habana} & $a_3$ & \'Algebra & 4\\
        $e_1$ & Juan & {111} & { La Habana} & $a_4$ & Programaci\'on & 5\\
        $e_3$ & Pedro & {111} & { La Habana} & $a_3$ & \'Algebra & 4\\
        $e_2$ & Mar\'ia & {112} & { Matanzas} & $a_1$ & An\'alisis & 3\\
        $e_2$ & Mar\'ia &  {112} & { Matanzas} & $a_2$ & L\'ogica & 3\\
        $e_4$ & Rita &  {112} & { Mayabeque} & $a_2$ & L\'ogica & 3\\
        $e_4$ & Rita &  {112} & { Mayabeque} & $a_4$ & Programaci\'on & 4\\
        $e_5$ & Carlos &  {113} & { Pinar del R\'io} & $a_3$ & \'Algebra & 3
    \end{tabular}
    \vspace{5mm}

    \centering
    \only<2,3>{
        {¿Se pudiera insertar un alumno que todav\'ia no ha recibido evaluaciones?}\\[2mm]
        \begin{tabular}{ccccccc}
            {\color<3>{red}$e_6$} & Marcos & 111 & La Habana & {\color<3>{red}NULL} & NULL & NULL
            
        \end{tabular}
        }
\end{frame}

\begin{frame}{¿Es este un buen dise\~no? (Anomal\'ia de eliminaci\'on)}
    \centering
    \begin{tabular}{ccccccc}
        \underline{\#E} & ENombre & Grupo & Provincia & \underline{\#A} & ANombre & Nota\\
        $e_1$ & Juan & 111 & La Habana & $a_1$ & An\'alisis & 3\\
        $e_1$ & Juan & 111 & La Habana & $a_2$ & L\'ogica & 2\\
        $e_1$ & Juan & 111 & La Habana  & $a_3$ & \'Algebra & 4\\
        $e_1$ & Juan & 111 & La Habana & $a_4$ & Programaci\'on & 5\\
        $e_3$ & Pedro & 111 & La Habana & $a_3$ & \'Algebra & 4\\
        $e_2$ & Mar\'ia & 112 &  Matanzas & $a_1$ & An\'alisis & 3\\
        $e_2$ & Mar\'ia &  112 & Matanzas & $a_2$ & L\'ogica & 3\\
        $e_4$ & Rita & 112 & Mayabeque & $a_2$ & L\'ogica & 3\\
        $e_4$ & Rita &  112 & Mayabeque & $a_4$ & Programaci\'on & 4\\
        \onslide<-1>{
        $e_5$ & Carlos &  113 & Pinar del R\'io & $a_3$ & \'Algebra & 3
        }
    \end{tabular}
    \vspace{5mm}

    \only<1>{
        ¿Qu\'e ocurre si se eliminan las notas del estudiante $e_5$?
    }
    \only<2>{
        \textcolor{red}{Se pierde la informaci\'on relacionada con la provincia Pinar del R\'io y el grupo C113}
    }

\end{frame}


\begin{frame}{¿Es este un buen dise\~no? (Anomal\'ia de modificaci\'on)}
    \centering
    \begin{tabular}{ccccccc}
        \underline{\#E} & ENombre & Grupo & Provincia & \underline{\#A} & ANombre & Nota\\
        $e_1$ & Juan & 111 & {\color<2>{red} La Habana} & $a_1$ & An\'alisis & 3\\
        $e_1$ & Juan & 111 & {\color<2>{red} La Habana} & $a_2$ & L\'ogica & 2\\
        $e_1$ & Juan & 111 & {\color<2>{red} La Habana} & $a_3$ & \'Algebra & 4\\
        $e_1$ & Juan & 111 & {\color<2>{red} La Habana} & $a_4$ & Programaci\'on & 5\\
        $e_3$ & Pedro & 111& La Habana & $a_3$ & \'Algebra & 4\\
        $e_2$ & Mar\'ia & 112 & Matanzas & $a_1$ & An\'alisis & 3\\
        $e_2$ & Mar\'ia &  112 &  Matanzas & $a_2$ & L\'ogica & 3\\
        $e_4$ & Rita &  112 & Mayabeque & $a_2$ & L\'ogica & 3\\
        $e_4$ & Rita &  112 & Mayabeque & $a_4$ & Programaci\'on & 4\\
        $e_5$ & Carlos &  113 & Pinar del R\'io & $a_3$ & \'Algebra & 3
    \end{tabular}
    \vspace{5mm}

    \only<1>{
        ¿Qu\'e tendr\'iamos que hacer si queremos cambiar la provincia de Juan?
    }
    \only<2>{
        \textcolor{red}{Todas las tuplas deben ser modificadas en una misma transacci\'on}
    }
\end{frame}

\begin{frame}
    \frametitle{?`Es este un buen dise\~no? (Redundacia)}

    \centering
    \begin{tabular}{ccccccc}
        \underline{\#E} & ENombre & Grupo & Provincia & \underline{\#A} & ANombre & Nota\\
        $e_1$ & Juan & {\color{orange}111} & {\color{orange} La Habana} & $a_1$ & An\'alisis & 3\\
        $e_1$ & Juan & {\color{orange}111} & {\color{orange} La Habana} & $a_2$ & L\'ogica & 2\\
        $e_1$ & Juan & {\color{orange}111} & {\color{orange} La Habana} & $a_3$ & \'Algebra & 4\\
        $e_1$ & Juan & {\color{orange}111} & {\color{orange} La Habana} & $a_4$ & Programaci\'on & 5\\
        $e_3$ & Pedro & {\color{orange}111} & {\color{orange} La Habana} & $a_3$ & \'Algebra & 4\\
        $e_2$ & Mar\'ia & {\color{blue}112} & {\color{blue} Matanzas} & $a_1$ & An\'alisis & 3\\
        $e_2$ & Mar\'ia &  {\color{blue}112} & {\color{blue} Matanzas} & $a_2$ & L\'ogica & 3\\
        $e_4$ & Rita &  {\color{blue}112} & {\color{blue} Mayabeque} & $a_2$ & L\'ogica & 3\\
        $e_4$ & Rita &  {\color{blue}112} & {\color{blue} Mayabeque} & $a_4$ & Programaci\'on & 4\\
        $e_5$ & Carlos &  {\color{green}113} & {\color{green} Pinar del R\'io} & $a_3$ & \'Algebra & 3
    \end{tabular}
    \vspace{5mm}

    \centering
    {\textcolor{red}{¿Es necesaria esta redundancia?}}
\end{frame}








\begin{frame}{Entonces...}
    \centering
    \Large ¿C\'omo solucionar estas anomal\'ias?
\end{frame}

\begin{frame}{F\'acil...}
    \begin{columns}[T]
        \begin{column}{0.68\linewidth}
            \begin{columns}[T]
                \begin{column}{0.6\textwidth}
                    \begin{center}
                        \textbf{Estudiante}\\[2mm]
        
                        \begin{tabular}{ccc}
                            \underline{\#E} & ENombre & Provincia\\[1mm]
                            \hline
                            $e_1$ & Juan & La Habana\\
                            $e_2$ & Mar\'ia & Matanzas\\
                            $e_3$ & Pedro & La Habana\\
                            $e_4$ & Rita & Mayabeque\\
                            $e_5$ & Carlos & Pinar del R\'io
                        \end{tabular}
                    \end{center}
                \end{column}

                \begin{column}{0.4\textwidth}
                    \begin{center}
                        \textbf{Provincia-Grupo}\\[2mm]
        
                        \begin{tabular}{cc}
                            \underline{Provincia} & Grupo\\[1mm]
                            \hline
                            La Habana & 111\\
                            Matanzas & 112 \\
                            Mayabeque & 112 \\
                            Pinar del R\'io & 113
                            
                        \end{tabular}
                    \end{center}
                \end{column}
                
            \end{columns}

            \begin{center}
                \textbf{Asignatura}\\[2mm]

                \begin{tabular}{cc}
                    \underline{\#A} & ANombre\\[1mm]
                    \hline
                    $a_1$ & An\'alisis\\
                    $a_2$ & L\'ogica \\
                    $a_3$ & \'Algebra\\
                    $a_4$ & Programaci\'on
                    
                \end{tabular}
            \end{center}
            
        \end{column}

        \begin{column}{0.3\linewidth}
            \vspace{6mm}
            \begin{center}
                \textbf{Evaluar}\\[2mm]

                \begin{tabular}{ccc}
                    \underline{\#E} & \underline{\#A} & Nota\\[1mm]
                    \hline
                    $e_1$ & $a_1$ & 3\\
                    $e_1$ & $a_2$ & 2\\
                    $e_1$ & $a_3$ & 4\\
                    $e_1$ & $a_4$ & 5\\
                    $e_3$ & $a_3$ & 4\\
                    $e_2$ & $a_1$ & 3\\
                    $e_2$ & $a_2$ & 3\\
                    $e_4$ & $a_2$ & 3\\
                    $e_4$ & $a_4$ & 4\\
                    $e_5$ & $a_3$ & 3\\
                \end{tabular}
            \end{center}
        \end{column}
    \end{columns}
\end{frame}

    \begin{frame}{Formalizando el dise\~no}
        \centering
        \Large{ ¿C\'omo obtener esta soluci\'on?}
        
        \vfill
        
        \hfill
        \begin{minipage}{0.3\textwidth}
            \only<2>{\it continuar\'a...}
        \end{minipage}
    \end{frame}
