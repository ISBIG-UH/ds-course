\begin{frame}
    \frametitle{Transacciones}

    \Huge \centering Transacciones

\end{frame}

\begin{frame}[fragile]
    \frametitle{Transacciones}

    Atomicidad, consistencia, aislamiento y durabilidad (ACID).

    \begin{block}<2->{Secuencia de comandos}
    \begin{enumerate}
        \item<3-> Use \textcolor{codepurple}{START TRANSACTION} para marcar el comienzo del bloque de c\'odigo.
        \item<4-> Ejecute una o m\'as operaciones (\textcolor{codepurple}{INSERT}, \textcolor{codepurple}{UPDATE}, \textcolor{codepurple}{DELETE}, etc).
        \item<5-> Use \textcolor{codepurple}{COMMIT} para aplicar los cambios o \textcolor{codepurple}{ROLLBACK} para deshacerlos.
    \end{enumerate}
    \end{block}

\end{frame}

\begin{frame}[fragile]
    \frametitle{Transacciones}
    \framesubtitle{Ejemplo}

	\begin{lstlisting}[ language=SQL,
		deletekeywords={IDENTITY},
		deletekeywords={[2]INT},
		morekeywords={clustered},
		framesep=8pt,
		xleftmargin=10pt,
		framexleftmargin=10pt,
		frame=tb,
		framerule=0pt,
        ]
START TRANSACTION;

-- Intentar deducir la cantidad de la cuenta del emisor
UPDATE Cuenta SET saldo = saldo - 100 WHERE Id = 1;
-- Verificar aquí si hay alguna condición de error

-- Intentar agregar la cantidad a la cuenta del receptor
UPDATE Cuenta SET saldo = saldo + 100 WHERE Id = 2;
-- Verificar aquí si hay alguna condición de error 

-- Manejo de errores
-- Aquí, estamos asumiendo que se utiliza una variable @error_detected para rastrear el estado del error
IF @error_detected THEN
    ROLLBACK;
ELSE
    COMMIT;
END IF;
\end{lstlisting}

\end{frame}