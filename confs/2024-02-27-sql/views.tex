\begin{frame}
    \frametitle{Reutilizando consultas}

    \begin{columns}
        \column{.4\textwidth}
    \centering
    \huge ?`Podemos reutilizar consultas?

        \column<2>{.55\textwidth}
        \includegraphics[width=\textwidth]{mysql-yeah.jpg}
    \end{columns}

\end{frame}

\begin{frame}[fragile]
    \frametitle{Vistas}

    Las vistas \textbf{son consultas almacenadas}. 

	\begin{lstlisting}[ language=SQL,
		deletekeywords={IDENTITY},
		deletekeywords={[2]INT},
		morekeywords={clustered},
		framesep=8pt,
		xleftmargin=40pt,
		framexleftmargin=40pt,
		frame=tb,
		framerule=0pt ]
CREATE VIEW <nombre_vista> AS
<consulta>
\end{lstlisting}

    \

    \only<2->{
    Puedes consultar a una vista de la misma manera que a una tabla.
    }

    \begin{columns}[t]
        \column<2>{.55\textwidth}
	\begin{lstlisting}[ language=SQL,
		deletekeywords={IDENTITY},
		deletekeywords={[2]INT},
		morekeywords={clustered},
		framesep=8pt,
		xleftmargin=10pt,
		framexleftmargin=10pt,
		frame=tb,
		framerule=0pt,
        caption={Creaci\'on de la vista.} ]
CREATE VIEW EmpleadosTecno AS
SELECT Nombre, Departamento
FROM Empleado
WHERE Departamento = "IT";
\end{lstlisting}

        \column<2>{.45\textwidth}
	\begin{lstlisting}[ language=SQL,
		deletekeywords={IDENTITY},
		deletekeywords={[2]INT},
		morekeywords={clustered},
		framesep=8pt,
		xleftmargin=10pt,
		framexleftmargin=10pt,
		frame=tb,
		framerule=0pt,
        caption={Consulta a la vista.} ]
SELECT Nombre
FROM EmpleadosTecno
WHERE Nombre LIKE "Mc%";
\end{lstlisting}
    \end{columns}

\end{frame}