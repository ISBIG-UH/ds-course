\begin{frame}
    \frametitle{Bases de datos}
    \framesubtitle{de Grafos}

\begin{center}
    \Huge \textbf{de Grafos}
\end{center}
\end{frame}

\begin{frame}
    \frametitle{?`Qu\'e es un grafo?}

    \begin{center}
        \includegraphics<2>[scale=.9]{img/graph-db/graph-vertices.pdf}
        \includegraphics<3>[scale=.9]{img/graph-db/graph-edges.pdf}
        \includegraphics<4>[scale=.9]{img/graph-db/graph-directed-edges.pdf}
    \end{center}

    \note<4>{@NOTE di casos de uso}
\end{frame}

\begin{frame}[fragile]{Bases de datos}
    \framesubtitle{de Grafos}

    \begin{overlayarea}{\linewidth}{\textheight}
        \begin{onlyenv}
            \begin{block}{}
                \begin{quote}
                    ``... una base de datos es una colecci\'on auto-descriptiva de \textcolor{red}{registros} integrados."
                    \hspace{1em plus 1fill}---Allen Taylor
                \end{quote}
                
                \textcolor{red}{Registro}: datos espec\'ificos sobre una entidad u objeto de inter\'es
            \end{block}
      \end{onlyenv}
      \vspace{2mm}
      \begin{center}
        
        % @TODO las etiketas de las imgs se ven fula en el edge web browser
        \includegraphics[scale=.77]{img/graph-db/graph-db-data.pdf}

      \end{center}
    \end{overlayarea}

    \note{@NOTE las interrelaciones tambi\'en pueden tener datos asociados}
\end{frame}

\begin{frame}[fragile]{Bases de datos}
    \framesubtitle{de Grafos}

    \begin{overlayarea}{\linewidth}{\textheight}
        \begin{onlyenv}
            \begin{block}{}
            \begin{quote}
                ``... una base de datos es una colecci\'on \textcolor{red}{auto-descriptiva} de registros integrados."
                \hspace{1em plus 1fill}---Allen Taylor
            \end{quote}
    
            \textcolor{red}{Auto-descriptiva}: se almacenan metadatos (la descripci\'on de su estructura) dentro
            del diccionario de datos de la propia base de datos.
        \end{block}
      \end{onlyenv}
      
          \vspace{5mm}
\centering

      \includegraphics[scale=.7]{img/graph-db/graph-db-auto-descriptive.pdf}

    \end{overlayarea}

    \note{@NOTE no se fuerza un esquema, pero las etiquetas y las llaves de las propiedades se encargan de describir los datos}
\end{frame}



\begin{frame}[fragile]{Bases de datos}
    \framesubtitle{de Grafos}

    \begin{overlayarea}{\linewidth}{\textheight}
        \begin{onlyenv}
            \begin{block}{}
                \begin{quote}
                    ``... una base de datos es una colecci\'on auto-descriptiva de registros \textcolor{red}{integrados}."
                    \hspace{1em plus 1fill}---Allen Taylor
                \end{quote}
                \textcolor{red}{Integrados}: no solo contiene los datos sino tambi\'en las interrelaciones
                 que se establecen entre estos.
            \end{block}
      \end{onlyenv}

      \vspace{5mm}
      
      \includegraphics[scale=.74]{img/graph-db/graph-db-relations.pdf}
    \end{overlayarea}
    
    \note{@NOTE las interrelaciones son ciudadanas de 1ra clase}
\end{frame}