\begin{frame}
    \frametitle{Bases de datos}
    \framesubtitle{Relacionales}

\begin{center}
    \Huge \textbf{Relacionales}
\end{center}
\end{frame}

\begin{frame}{Bases de datos}
    \framesubtitle{Relacionales}

    \begin{overlayarea}{\linewidth}{\textheight}
        \begin{onlyenv}
            \begin{block}{}
                \begin{quote}
                    ``... una base de datos es una colecci\'on auto-descriptiva de \textcolor{red}{registros} integrados."
                    \hspace{1em plus 1fill}---Allen Taylor
                \end{quote}
                
                \textcolor{red}{Registro}: datos espec\'ificos sobre una entidad u objeto de inter\'es
            \end{block}
      \end{onlyenv}
      \vspace{8mm}
      \centering
      \begin{tabular}{|c|c|c|c|c|}
          \hline
          23082300205 & Edgar F. & Codd & 23/08/1923\\
          \hline
      \end{tabular}
    \end{overlayarea}
\end{frame}

\begin{frame}{Bases de datos}
    \framesubtitle{Relacionales}

    \begin{overlayarea}{\linewidth}{\textheight}
        \begin{onlyenv}
            \begin{block}{}
            \begin{quote}
                ``... una base de datos es una colecci\'on \textcolor{red}{auto-descriptiva} de registros integrados."
                \hspace{1em plus 1fill}---Allen Taylor
            \end{quote}
    
            \textcolor{red}{Auto-descriptiva}: se almacenan metadatos (la descripci\'on de su estructura) dentro
            del diccionario de datos de la propia base de datos.
        \end{block}
      \end{onlyenv}
      
          \vspace{5mm}
          \large \textbf{Persona}
          \vspace{2mm}
          \centering{
          
      
          \begin{tabular}{|c|c|c|c|}
              \hline
              CI : \textcolor{blue}{\texttt{string}} & Nombre : \textcolor{blue}{\texttt{string}} & Apellido : \textcolor{blue}{\texttt{string}} & F. Nacimiento : \textcolor{blue}{\texttt{date}} \\ 
              \hline
            23082300205 & Edgar F. & Codd & 23/08/1923\\
              \hline
          \end{tabular}
          }
    \end{overlayarea}
\end{frame}



\begin{frame}{Bases de datos}
    \framesubtitle{Relacionales}

    \begin{overlayarea}{\linewidth}{\textheight}
        \begin{onlyenv}
            \begin{block}{}
                \begin{quote}
                    ``... una base de datos es una colecci\'on auto-descriptiva de registros \textcolor{red}{integrados}."
                    \hspace{1em plus 1fill}---Allen Taylor
                \end{quote}
                \textcolor{red}{Integrados}: no solo contiene los datos sino tambi\'en las interrelaciones
                 que se establecen entre estos.
            \end{block}
      \end{onlyenv}

      \vspace{5mm}
    \large \textbf{Persona}
    \vspace{2mm}
    \centering{
    

    \begin{tabular}{|c|c|c|c|}
        \hline
        CI : \textcolor{blue}{\texttt{string}} & Nombre : \textcolor{blue}{\texttt{string}} & Apellido : \textcolor{blue}{\texttt{string}} & F. Nacimiento : \textcolor{blue}{\texttt{date}} \\ 
        \hline
        \textcolor{red}{23082300205} & Edgar F. & Codd & 23/08/1923\\
        \hline
    \end{tabular}
    }

    \vspace{5mm}


    \large \textbf{Cuenta}

    \vspace{2mm}
    \centering{
    \begin{tabular}{|c|c|c|}
        \hline
        No. Cuenta : \textcolor{blue}{\texttt{string}} & Balance : \textcolor{blue}{\texttt{decimal}} & CI Due\~no : \textcolor{blue}{\texttt{string}} \\
        \hline
        8976 & 270.98 & \textcolor{red}{23082300205} \\
        \hline
    \end{tabular}
    }
    \end{overlayarea}
    

    \note{@NOTE todo esto permite garantizar la consistencia en todo momento, lo cual es la principal fortaleza de las bds relacionales}
\end{frame}

