\documentclass[aspectratio=169]{beamer}
\usepackage{tikz}
\usepackage{graphicx}
\usepackage[spanish]{babel}

\usetikzlibrary{shapes.geometric}
\usetikzlibrary{shapes.arrows}
\usetikzlibrary{overlay-beamer-styles}
\setbeamertemplate{navigation symbols}{}

\definecolor{blue1}{RGB}{126,126,206}
\definecolor{blue2}{RGB}{87,87,192}
\definecolor{blue3}{RGB}{51,51,178}
\definecolor{blue4}{RGB}{27,26,107}


\usepackage[
    backend=biber,
    sorting=ynt]{biblatex}
\addbibresource{biblio.bib}

% \usetheme{Warsaw}
\usecolortheme{whale}
\title{Bases de Datos}
\subtitle{Introducci\'on a las Bases de Datos}

\author[Garc\'ia L., Cardentey V. M.]{ Lic. V\'ictor M. Cardentey Fundora \\ Dra. Lucina Garc\'ia Hern\'andez}

\institute[MATCOM-UH]{
    Departamento de Computaci\'on\\
    Facultad de Matem\'atica y Computaci\'on\\
    Universidad de La Habana
}

\date[]{20 de febrero de 2023}
\begin{document}
\maketitle
\begin{frame}{¿Qu\'e tan grande es YouTube?}

    \begin{columns}[T]
        \begin{column}{0.48\textwidth}
            \begin{overlayarea}{\linewidth}{\textheight}
                \vspace{17mm}

                \begin{itemize}
                    \item<1-> 500 horas de video subidas cada minuto
                    \item<2-> 720000 horas de video subidas cada d\'ia
                    \item<3-> Se necesitan \textcolor{red}{82 a\~nos} para ver el contenido de un solo d\'ia
                \end{itemize}
            \end{overlayarea}

        \end{column}

        \begin{column}{0.48\textwidth}
            \vspace{17mm}
            
            \includegraphics<3->[width=\textwidth]{img/skelly.jpg}
        \end{column}
    \end{columns}
\end{frame}

{
\setbeamertemplate{background} 
{
    \includegraphics[width=\paperwidth,height=\paperheight]{img/Big-tech-banking-LatAm.jpg}
}
\begin{frame}
\end{frame}
}

\begin{frame}{Motivaci\'on}
    \begin{block}{Problemas a resolver}
        \begin{itemize}
            \item<2-> Garantizar la persistencia de los datos generados por aplicaciones y dispositivos.
            \item<3-> Utilizar grandes cantidades de datos de forma eficiente.
        \end{itemize}
    \end{block}
    
    \vspace{10pt}

    \centering
    \begin{tikzpicture}
        \onslide<2->{
        \node at (1.4,6.4) {\tiny \bf {Apps Web}} ;
        \node[inner sep=0pt] (web) at (1.4,5.6) {
            \includegraphics[width=1.3cm]{img/web.png}
        };

        \node at (-0.1, 5.5) {\tiny \bf {Apps Escritorio}};
        \node[inner sep=0pt] (desktop) at (0,4.7){
            \includegraphics[width=1.3cm]{img/desktop.png}
        };

        \node at (0,2.6) {\tiny \bf {Apps m\'oviles}};
        \node[inner sep=0pt] (mobile) at (0,3.4){
            \includegraphics[width=1.3cm]{img/mobile.png}
        };

        \node at (1.4,1.7) {\tiny \bf {Internet de las cosas}};
        \node[inner sep=0pt] (iot) at (1.4,2.5) {
            \includegraphics[width=1.3cm]{img/iot.png}
        };

        \node at (4,5) {\tiny \bf {Datos}};
        \node[inner sep=0pt] (database) at (4,4) {
            \includegraphics[width=1.8cm]{img/data.png}
        };

        }

        \onslide<2>{
            \draw[->,thick] (web.south east) -- (3.1,4.3);
            \draw[->,thick] (desktop.east) -- (3.1,4.1);
            \draw[->,thick] (mobile.east) -- (3.1,3.8);
            \draw[->,thick] (iot.north east) -- (3.1,3.6);
        }

        \onslide<3->{
            \draw[<->,thick] (web.south east) -- (3.1,4.3);
        \draw[<->,thick] (desktop.east) -- (3.1,4.1);
        \draw[<->,thick] (mobile.east) -- (3.1,3.8);
        \draw[<->,thick] (iot.north east) -- (3.1,3.6);

        \node at (6.5,6.4) {\tiny \bf {Reportes}};
        \node[inner sep=0pt] (report) at (6.5, 5.6) {
            \includegraphics[width=1.3cm]{img/report.png}
        };
        \node at (7.9,4.9) {\tiny \bf {Modelos de IA}};
        \node[inner sep=0pt] (ai) at (7.9, 4.1) {
            \includegraphics[width=1.3cm]{img/ia.png}
        };
        \node at (6.5,1.7) {\tiny \bf {Recomendaciones}};
        \node[inner sep=0pt] (ranking) at (6.5, 2.5) {
            \includegraphics[width=1.3cm]{img/ranking.png}
        };
        
        \draw[->,thick] (4.9,4.3) -- (report.south west);
        \draw[->,thick] (database.east) -- (ai.west);
        \draw[->,thick] (4.9,3.6) -- (ranking.north west);
        }
        
        
    \end{tikzpicture}
  
  
\end{frame}

\begin{frame}{¿C\'omo?}
    \begin{block}<2>{}
        La forma m\'as sencilla de almacenar datos es escribirlos en un fichero
    \end{block}
    \vspace{5mm}

    \centering
    \includegraphics<2>[width=50mm, height=50mm]{img/csv.png}
\end{frame}


\begin{frame}{Sistemas orientados a ficheros}
   
    \begin{columns}[T]
        \begin{column}{0.30\linewidth}
            \begin{block}
                
                \begin{tabular}{c c c}
    
                    \includegraphics[width=1cm]{img/list.png}
                    
                    & \includegraphics[width=1cm]{img/list.png}
                    & \includegraphics[width=1cm]{img/list.png}\\
    
                    \includegraphics[width=1cm]{img/list.png}
                    & \includegraphics[width=1cm]{img/list.png}
                    & \includegraphics[width=1cm]{img/list.png}\\
    
                    \includegraphics[width=1cm]{img/list.png}
                    & \includegraphics[width=1cm]{img/list.png}
                    & \includegraphics[width=1cm]{img/list.png}\\
    
    
                    ... & ... & ...\\
    
                    \includegraphics[width=1cm]{img/list.png}
                    & \includegraphics[width=1cm]{img/list.png}
                    & \includegraphics[width=1cm]{img/list.png}\\
    
                    \includegraphics[width=1cm]{img/list.png}
                    & \includegraphics[width=1cm]{img/list.png}
                    & \includegraphics[width=1cm]{img/list.png}\\
    
                \end{tabular}
            \end{block}
            
    \end{column}

    


        \begin{column}{0.68\linewidth}
            \begin{block}{Caracter\'isticas}
                \begin{itemize}
                    \item Se crean nuevos ficheros a medida que se crean nuevos tipos de registros o se terminan los ficheros.
                    \item Cada fichero se opera de forma independiente del resto de archivos en almacenamiento.
                \end{itemize}
                 
            \end{block}
            \begin{alertblock}<2->{Limitaciones}
                \begin{itemize}
                    \item<2-> Baja eficiencia
                    \item<3-> Gran redundancia de los datos
                    \item<4-> Pobre control sobre los datos
                    \item<5-> Capacidades inadecuadas de manipulaci\'on de datos
                \end{itemize}
            \end{alertblock}
        \end{column}

    \end{columns}
\end{frame}
\begin{frame}{Bases de datos}
    \begin{overlayarea}{\linewidth}{\textheight}
            \begin{block}{}
                \begin{quote}
                    ``... una base de datos es una colecci\'on \textcolor<3>{blue}{auto-descriptiva} de \textcolor<2>{blue}{registros} \textcolor<4>{blue}{integrados}."
                    \hspace{1em plus 1fill}---Allen Taylor
                \end{quote}
            \end{block}
                
    \begin{block}<2->{Registro}
        Datos espec\'ificos sobre una entidad u objeto de inter\'es.
    \end{block}

    \begin{block}<3->{Auto-descriptiva}
        Se almacenan metadatos (la descripci\'on de su estructura) dentro.
    \end{block}

    \begin{block}<4->{Integrados}
        No solo contiene los datos sino tambi\'en las interrelaciones.
    \end{block}
    \end{overlayarea}

    \pause
   
   \note<5>{@NOTE Ahora vamos a ir explicando c\'omo se ajustan los t\'erminos \textit{auto-descriptiva}, \textit{registros} e \textit{integrados} a cada tipo de BDs que vamos a estudiar}
\end{frame}

\begin{frame}
    \frametitle{Bases de datos}
    \framesubtitle{Relacionales}

\begin{center}
    \Huge \textbf{Relacionales}
\end{center}
\end{frame}

\begin{frame}{Bases de datos}
    \framesubtitle{Relacionales}

    \begin{overlayarea}{\linewidth}{\textheight}
        \begin{onlyenv}
            \begin{block}{}
                \begin{quote}
                    ``... una base de datos es una colecci\'on auto-descriptiva de \textcolor{red}{registros} integrados."
                    \hspace{1em plus 1fill}---Allen Taylor
                \end{quote}
                
                \textcolor{red}{Registro}: datos espec\'ificos sobre una entidad u objeto de inter\'es
            \end{block}
      \end{onlyenv}
      \vspace{8mm}
      \centering
      \begin{tabular}{|c|c|c|c|c|}
          \hline
          23082300205 & Edgar F. & Codd & 23/08/1923\\
          \hline
      \end{tabular}
    \end{overlayarea}
\end{frame}

\begin{frame}{Bases de datos}
    \framesubtitle{Relacionales}

    \begin{overlayarea}{\linewidth}{\textheight}
        \begin{onlyenv}
            \begin{block}{}
            \begin{quote}
                ``... una base de datos es una colecci\'on \textcolor{red}{auto-descriptiva} de registros integrados."
                \hspace{1em plus 1fill}---Allen Taylor
            \end{quote}
    
            \textcolor{red}{Auto-descriptiva}: se almacenan metadatos (la descripci\'on de su estructura) dentro
            del diccionario de datos de la propia base de datos.
        \end{block}
      \end{onlyenv}
      
          \vspace{5mm}
          \large \textbf{Persona}
          \vspace{2mm}
          \centering{
          
      
          \begin{tabular}{|c|c|c|c|}
              \hline
              CI : \textcolor{blue}{\texttt{string}} & Nombre : \textcolor{blue}{\texttt{string}} & Apellido : \textcolor{blue}{\texttt{string}} & F. Nacimiento : \textcolor{blue}{\texttt{date}} \\ 
              \hline
            23082300205 & Edgar F. & Codd & 23/08/1923\\
              \hline
          \end{tabular}
          }
    \end{overlayarea}
\end{frame}



\begin{frame}{Bases de datos}
    \framesubtitle{Relacionales}

    \begin{overlayarea}{\linewidth}{\textheight}
        \begin{onlyenv}
            \begin{block}{}
                \begin{quote}
                    ``... una base de datos es una colecci\'on auto-descriptiva de registros \textcolor{red}{integrados}."
                    \hspace{1em plus 1fill}---Allen Taylor
                \end{quote}
                \textcolor{red}{Integrados}: no solo contiene los datos sino tambi\'en las interrelaciones
                 que se establecen entre estos.
            \end{block}
      \end{onlyenv}

      \vspace{5mm}
    \large \textbf{Persona}
    \vspace{2mm}
    \centering{
    

    \begin{tabular}{|c|c|c|c|}
        \hline
        CI : \textcolor{blue}{\texttt{string}} & Nombre : \textcolor{blue}{\texttt{string}} & Apellido : \textcolor{blue}{\texttt{string}} & F. Nacimiento : \textcolor{blue}{\texttt{date}} \\ 
        \hline
        \textcolor{red}{23082300205} & Edgar F. & Codd & 23/08/1923\\
        \hline
    \end{tabular}
    }

    \vspace{5mm}


    \large \textbf{Cuenta}

    \vspace{2mm}
    \centering{
    \begin{tabular}{|c|c|c|}
        \hline
        No. Cuenta : \textcolor{blue}{\texttt{string}} & Balance : \textcolor{blue}{\texttt{decimal}} & CI Due\~no : \textcolor{blue}{\texttt{string}} \\
        \hline
        8976 & 270.98 & \textcolor{red}{23082300205} \\
        \hline
    \end{tabular}
    }
    \end{overlayarea}
    

    \note{@NOTE todo esto permite garantizar la consistencia en todo momento, lo cual es la principal fortaleza de las bds relacionales}
\end{frame}


\begin{frame}
    \frametitle{Bases de datos}
    \framesubtitle{Llave-valor}

\begin{center}
    \Huge \textbf{Llave-valor}
\end{center}
\end{frame}

\begin{frame}[fragile]{Bases de datos}
    \framesubtitle{Llave-valor}

    \begin{overlayarea}{\linewidth}{\textheight}
        \begin{onlyenv}
            \begin{block}{}
                \begin{quote}
                    ``... una base de datos es una colecci\'on auto-descriptiva de \textcolor{red}{registros} integrados."
                    \hspace{1em plus 1fill}---Allen Taylor
                \end{quote}
                
                \textcolor{red}{Registro}: datos espec\'ificos sobre una entidad u objeto de inter\'es
            \end{block}
      \end{onlyenv}
      \vspace{2mm}
      \begin{center}
        
\begin{minipage}{0.6\textwidth}
    \begin{onlyenv}<1>
\begin{lstlisting}[language=text]
`persona_23082300205': `{  
    "CI": 23082300205, 
    "Nombre": "Edgar F.", 
    "Apellido": "Codd", 
    "F. Nacimiento": 23/08/1923 
}'
\end{lstlisting}
    \end{onlyenv}

    \begin{onlyenv}<2>
\begin{lstlisting}[language=text]
`#persona_23082300205#': `{  
    "CI": 23082300205, 
    "Nombre": "Edgar F.", 
    "Apellido": "Codd", 
    "F. Nacimiento": 23/08/1923 
}'
\end{lstlisting}
    \end{onlyenv}

\begin{onlyenv}<3>
\begin{lstlisting}[language=text]
`persona_23082300205': `#{  
    "CI": 23082300205, 
    "Nombre": "Edgar F.", 
    "Apellido": "Codd", 
    "F. Nacimiento": 23/08/1923 
}#'
\end{lstlisting}
\end{onlyenv}

\end{minipage}
      \end{center}
    \end{overlayarea}

    \note<1>{@NOTE el objetivo de estas bds es garantizar el acceso eficiente al valor asociado a una llave. Vamos a ver m\'as adelante c\'omo esto se logra}
    \note<2>{@NOTE la llave}
    \note<3>{@NOTE el valor \textbf{no se interpreta, no se asume estructurado, se almacena tal cual}. Se espera un \textit{blob} o string}
\end{frame}

\begin{frame}[fragile]{Bases de datos}
    \framesubtitle{Llave-valor}

    \begin{overlayarea}{\linewidth}{\textheight}
        \begin{onlyenv}
            \begin{block}{}
            \begin{quote}
                ``... una base de datos es una colecci\'on \textcolor{red}{auto-descriptiva} de registros integrados."
                \hspace{1em plus 1fill}---Allen Taylor
            \end{quote}
    
            \textcolor{red}{Auto-descriptiva}: se almacenan metadatos (la descripci\'on de su estructura) dentro
            del diccionario de datos de la propia base de datos.
        \end{block}
      \end{onlyenv}
      
          \vspace{5mm}
\centering

\begin{minipage}{0.6\textwidth}
\begin{lstlisting}[language=text]
`persona_23082300205': `{  
    "#CI#": 23082300205, 
    "#Nombre#": "Edgar F.", 
    "#Apellido#": "Codd", 
    "#F. Nacimiento#": 23/08/1923 
}'
\end{lstlisting}

\end{minipage}
    \end{overlayarea}

                \note{@NOTE no hay esquema. Mayor libertad. La integridad debe ser garantizada a nivel de app}
\end{frame}



\begin{frame}[fragile]{Bases de datos}
    \framesubtitle{Llave-valor}

    \begin{overlayarea}{\linewidth}{\textheight}
        \begin{onlyenv}
            \begin{block}{}
                \begin{quote}
                    ``... una base de datos es una colecci\'on auto-descriptiva de registros \textcolor{red}{integrados}."
                    \hspace{1em plus 1fill}---Allen Taylor
                \end{quote}
                \textcolor{red}{Integrados}: no solo contiene los datos sino tambi\'en las interrelaciones
                 que se establecen entre estos.
            \end{block}
      \end{onlyenv}

      \vspace{5mm}
        \begin{columns}
            \begin{column}[t]{0.5\textwidth}
                \centering

                \begin{lstlisting}[language=text]
`persona_#23082300205#': `{  
    "CI": 23082300205, 
    "Nombre": "Edgar F.", 
    "Apellido": "Codd", 
    "F. Nacimiento": 23/08/1923 
}'
                \end{lstlisting} 
            \end{column}

            \begin{column}[t]{0.5\textwidth}
                \centering

                \begin{lstlisting}[language=text]
`cuenta_8976': `{  
    "No. Cuenta": 8976, 
    "Balance": 270.98, 
    "CI Duenyo": #23082300205#
}'
                \end{lstlisting} 
            \end{column}
        \end{columns}
    \end{overlayarea}
    
    \note{@NOTE mantener la consistencia e integridad va a nivel de app. No es el objetivo principal de las BDs llave-valor}
\end{frame}

\begin{frame}
    \frametitle{Bases de datos}
    \framesubtitle{Columnares}

\begin{center}
    \Huge \textbf{Columnares}
\end{center}
\end{frame}

\begin{frame}{Bases de datos}
    \framesubtitle{Columnares}

    \begin{overlayarea}{\linewidth}{\textheight}
            \begin{block}{}
                \begin{quote}
                    ``... una base de datos es una colecci\'on auto-descriptiva de \textcolor{red}{registros} integrados."
                    \hspace{1em plus 1fill}---Allen Taylor
                \end{quote}
                
                \textcolor{red}{Registro}: datos espec\'ificos sobre una entidad u objeto de inter\'es
            \end{block}
      \vspace{8mm}

      \centering

        \begin{onlyenv}<1>
      \begin{tabular}{|c|}
          \hline
           23082300205 \\ \hline
           Edgar F. \\\hline
           Codd \\\hline
           23/08/1923 \\
          \hline
      \end{tabular}
    \end{onlyenv}
     
     \begin{onlyenv}<2->
      \begin{tabular}{|c|c|c|}
          \hline
           23082300205 & 43101123919 & 44011279165 \\ \hline
           Edgar F. & Michael & Jim \\\hline
           Codd & Stonebreaker & Gray \\\hline
           23/08/1923 & 11/10/1943 & 12/01/1944 \\
          \hline
      \end{tabular}
     \end{onlyenv}

    \end{overlayarea}

    \note<1>{@NOTE ahora el enfoque est\'a sobre las columnas, en lugar de las filas, i.e. es m\'as importante extraer todos los valores de una columna que todos los de una fila. Aqu\'i el principal objetivo es analizar comportamientos a partir de los datos}    
\end{frame}

\begin{frame}{Bases de datos}
    \framesubtitle{Columnares}

    \begin{overlayarea}{\linewidth}{\textheight}
        \begin{onlyenv}
            \begin{block}{}
            \begin{quote}
                ``... una base de datos es una colecci\'on \textcolor{red}{auto-descriptiva} de registros integrados."
                \hspace{1em plus 1fill}---Allen Taylor
            \end{quote}
    
            \textcolor{red}{Auto-descriptiva}: se almacenan metadatos (la descripci\'on de su estructura) dentro
            del diccionario de datos de la propia base de datos.
        \end{block}
      \end{onlyenv}
      
          \vspace{5mm}
      
        \begin{columns}
            \begin{column}[t]{0.5\textwidth}
          \centering

          \large \textbf{Persona}
          \vspace{2mm}

      \begin{tabular}{|c|c|}
          \hline
          CI : \textcolor{blue}{string} & 23082300205 \\ \hline
          Nombre : \textcolor{blue}{string} & Edgar F. \\\hline
          Apellido : \textcolor{blue}{string} & Codd \\\hline
          F. Nacimiento : \textcolor{blue}{date} & 23/08/1923 \\
          \hline
      \end{tabular}
    \end{column}
            \begin{column}[t]{0.5\textwidth}
            \end{column}
        \end{columns}

    \end{overlayarea}
\end{frame}



\begin{frame}{Bases de datos}
    \framesubtitle{Columnares}

    \begin{overlayarea}{\linewidth}{\textheight}
        \begin{onlyenv}
            \begin{block}{}
                \begin{quote}
                    ``... una base de datos es una colecci\'on auto-descriptiva de registros \textcolor{red}{integrados}."
                    \hspace{1em plus 1fill}---Allen Taylor
                \end{quote}
                \textcolor{red}{Integrados}: no solo contiene los datos sino tambi\'en las interrelaciones
                 que se establecen entre estos.
            \end{block}
      \end{onlyenv}

      \vspace{5mm}
        \begin{columns}
            \begin{column}[t]{0.5\textwidth}
          \centering

          \large \textbf{Persona}
          \vspace{2mm}

      \begin{tabular}{|c|c|}
          \hline
          CI : \textcolor{blue}{string} & \textcolor{red}{23082300205} \\ \hline
          Nombre : \textcolor{blue}{string} & Edgar F. \\\hline
          Apellido : \textcolor{blue}{string} & Codd \\\hline
          F. Nacimiento : \textcolor{blue}{date} & 23/08/1923 \\
          \hline
      \end{tabular}
    \end{column}

            \begin{column}[t]{0.5\textwidth}
          \centering

          \large \textbf{Cuenta}
          \vspace{2mm}

                \begin{tabular}{|c|c|}
                    \hline
                    No. Cuenta : \textcolor{blue}{string} & 8976 \\\hline
                    Balance : \textcolor{blue}{decimal} & 270.98 \\\hline
                    CI Due\~no : \textcolor{blue}{string} & \textcolor{red}{23082300205} \\
                    \hline
                \end{tabular}
            \end{column}
        \end{columns}
    \end{overlayarea}
    

    
\end{frame}

\begin{frame}
    \frametitle{Bases de datos}
    \framesubtitle{de Documentos}

\begin{center}
    \Huge \textbf{de Documentos}
\end{center}
\end{frame}

\begin{frame}[fragile]{Bases de datos}
    \framesubtitle{de Documentos}

    \begin{overlayarea}{\linewidth}{\textheight}
        \begin{onlyenv}
            \begin{block}{}
                \begin{quote}
                    ``... una base de datos es una colecci\'on \textcolor<2>{red}{auto-descriptiva} de \textcolor<1>{red}{registros} integrados."
                    \hspace{1em plus 1fill}---Allen Taylor
                \end{quote}
            \end{block}
      \end{onlyenv}

      \vspace{5mm}

      \begin{center}
        
        \begin{minipage}{.6\textwidth}
            \begin{onlyenv}<1>
        \begin{lstlisting}[language=json]
{  
    "CI": "23082300205", 
    "Nombre": "Edgar F.", 
    "Apellido": "Codd", 
    "F. Nacimiento": "23/08/1923",
    "Titulo de Doctor": {
        "F. Expedido": 1965,
        ...
    }
}
        \end{lstlisting}
            \end{onlyenv}

\begin{onlyenv}<2>
        \begin{lstlisting}[language=json]
{  
    "#CI#": "23082300205", 
    "#Nombre#": "Edgar F.", 
    "#Apellido#": "Codd", 
    "#F. Nacimiento#": "23/08/1923",
    "#Titulo de Doctor#": {
        "#F. Expedido#": 1965,
        ...
    }
}
        \end{lstlisting}
\end{onlyenv}
        \end{minipage}
      \end{center}
    \end{overlayarea}

    \note<1>{@NOTE han trabajado con json antes?}
    \note<2>{@NOTE el esquema puede variar entre docs de la misma colecci\'on, pero la estructura interna de un doc se encuentra bien definida}
\end{frame}

\begin{frame}[fragile]{Bases de datos}
    \framesubtitle{de Documentos}

    \begin{overlayarea}{\linewidth}{\textheight}
        \begin{onlyenv}
            \begin{block}{}
                \begin{quote}
                    ``... una base de datos es una colecci\'on auto-descriptiva de registros \textcolor{red}{integrados}."
                    \hspace{1em plus 1fill}---Allen Taylor
                \end{quote}
                \textcolor{red}{Integrados}: no solo contiene los datos sino tambi\'en las interrelaciones
                 que se establecen entre estos.
            \end{block}
      \end{onlyenv}

      \vspace{3mm}
        \begin{columns}
            \begin{column}[t]{0.525\textwidth}
                \centering
                \large \textbf{Persona} (llave:  ``\textcolor{blue}{23082300205}'')

        \begin{lstlisting}[language=json]
{  
    "CI": "23082300205", 
    "Nombre": "Edgar F.", 
    "Apellido": "Codd", 
    "F. Nacimiento": "23/08/1923"
}
        \end{lstlisting}
            \end{column}

            \begin{column}[t]{0.475\textwidth}
                \centering
                \large \textbf{Cuenta} (llave: ``8976'')

        \begin{lstlisting}[language=json]
{  
    "No. Cuenta": "8976", 
    "Balance": 270.98, 
    "CI Duenyo": "#23082300205#"
}
                \end{lstlisting} 
            \end{column}
        \end{columns}
    \end{overlayarea}
    
    \note{@NOTE dentro de un doc puede haber otro doc. Esta es otra manera de expresar una interrelaci\'on entre ellos}
\end{frame}

\begin{frame}
    \frametitle{Bases de datos}
    \framesubtitle{de Grafos}

\begin{center}
    \Huge \textbf{de Grafos}
\end{center}
\end{frame}

\begin{frame}
    \frametitle{?`Qu\'e es un grafo?}

    \begin{center}
        \includegraphics<2>[scale=.9]{img/graph-db/graph-vertices.pdf}
        \includegraphics<3>[scale=.9]{img/graph-db/graph-edges.pdf}
        \includegraphics<4>[scale=.9]{img/graph-db/graph-directed-edges.pdf}
    \end{center}

    \note<4>{@NOTE di casos de uso}
\end{frame}

\begin{frame}[fragile]{Bases de datos}
    \framesubtitle{de Grafos}

    \begin{overlayarea}{\linewidth}{\textheight}
        \begin{onlyenv}
            \begin{block}{}
                \begin{quote}
                    ``... una base de datos es una colecci\'on auto-descriptiva de \textcolor{red}{registros} integrados."
                    \hspace{1em plus 1fill}---Allen Taylor
                \end{quote}
                
                \textcolor{red}{Registro}: datos espec\'ificos sobre una entidad u objeto de inter\'es
            \end{block}
      \end{onlyenv}
      \vspace{2mm}
      \begin{center}
        
        % @TODO las etiketas de las imgs se ven fula en el edge web browser
        \includegraphics[scale=.77]{img/graph-db/graph-db-data.pdf}

      \end{center}
    \end{overlayarea}

    \note{@NOTE las interrelaciones tambi\'en pueden tener datos asociados}
\end{frame}

\begin{frame}[fragile]{Bases de datos}
    \framesubtitle{de Grafos}

    \begin{overlayarea}{\linewidth}{\textheight}
        \begin{onlyenv}
            \begin{block}{}
            \begin{quote}
                ``... una base de datos es una colecci\'on \textcolor{red}{auto-descriptiva} de registros integrados."
                \hspace{1em plus 1fill}---Allen Taylor
            \end{quote}
    
            \textcolor{red}{Auto-descriptiva}: se almacenan metadatos (la descripci\'on de su estructura) dentro
            del diccionario de datos de la propia base de datos.
        \end{block}
      \end{onlyenv}
      
          \vspace{5mm}
\centering

      \includegraphics[scale=.7]{img/graph-db/graph-db-auto-descriptive.pdf}

    \end{overlayarea}

    \note{@NOTE no se fuerza un esquema, pero las etiquetas y las llaves de las propiedades se encargan de describir los datos}
\end{frame}



\begin{frame}[fragile]{Bases de datos}
    \framesubtitle{de Grafos}

    \begin{overlayarea}{\linewidth}{\textheight}
        \begin{onlyenv}
            \begin{block}{}
                \begin{quote}
                    ``... una base de datos es una colecci\'on auto-descriptiva de registros \textcolor{red}{integrados}."
                    \hspace{1em plus 1fill}---Allen Taylor
                \end{quote}
                \textcolor{red}{Integrados}: no solo contiene los datos sino tambi\'en las interrelaciones
                 que se establecen entre estos.
            \end{block}
      \end{onlyenv}

      \vspace{5mm}
      
      \includegraphics[scale=.74]{img/graph-db/graph-db-relations.pdf}
    \end{overlayarea}
    
    \note{@NOTE las interrelaciones son ciudadanas de 1ra clase}
\end{frame}

\begin{frame}{Un cambio de enfoque}
    \begin{overlayarea}{\linewidth}{\textheight}
        \begin{onlyenv}
            \begin{block}{}
                
                En los sistemas orientados a ficheros los humanos tienen el control sobre los ficheros  
            \end{block}
        \end{onlyenv}

        \vspace{10mm}

        \centering
        \includegraphics<2>[height=50mm, width=80mm]{img/filesystem.png}
    \end{overlayarea}
\end{frame}


\begin{frame}{Un cambio de enfoque}
    \centering
    \begin{tikzpicture}
        \node at (0,3.7) {\tiny \bf {Usuario}} ;
        \node[inner sep=0pt] (user) at (0,3) {
            \includegraphics[width=1.3cm]{img/user.png}
        };


        \node at (6,3.7) {\tiny \bf {Fichero}} ;
        \node[inner sep=0pt] (file) at (6,3) {
            \includegraphics[width=1.3cm]{img/file.png}
        };

        \draw[<->,thick] (user.east) -- (file.west);
    \end{tikzpicture}

    El usuario interact\'ua directamente con los ficheros

    \vspace{20pt}
    \centering
    \begin{tikzpicture}
        \node at (0,3.7) {\tiny \bf {Usuario}} ;
        \node[inner sep=0pt] (user) at (0,3) {
            \includegraphics[width=1.3cm]{img/user.png}
        };

        \node at (3,3.7) {\tiny \bf {Software}} ;
        \node[inner sep=0pt] (dbms) at (3,3) {
            \includegraphics[width=1.3cm]{img/dbms.png}
        };

        \node at (6,3.7) {\tiny \bf {Base de datos}} ;
        \node[inner sep=0pt] (database) at (6,3) {
            \includegraphics[width=1.3cm]{img/database.png}
        };

        \draw[<->,thick] (user.east) -- (dbms.west);
        \draw[<->,thick] (dbms.east) -- (database.west);
    \end{tikzpicture}
    
    El usuario interact\'ua con la base de datos mediante
    \textit{software}
   
\end{frame}
\begin{frame}{¿Qu\'e es este \textit{software}?}



    \begin{block}{Sistemas de gesti\'on de bases de datos (SGBD)}
        \begin{quote}
            ``Un \textcolor{red}{sistema de gesti\'on de bases de datos}
            es un sistema computacional que proporciona
            \textcolor{red}{funcionalidades, medios o servicios para manipular} y, en particular,
            \textcolor{red}{manejar todos los accesos} a una base de datos o una colecci\'on
            de bases de datos." \hspace{1em plus 1fill}---C. J. Date
        \end{quote}
    \end{block}
\end{frame}


\begin{frame}{Superando limitaciones}
       
            \begin{block}{}
                \begin{itemize}
                    \item<1-> \textbf{Persistentes}: los datos permanecen en memoria externa
                    \item<2-> \textbf{Masivos}: manejan terabytes/petabytes de datos
                    \item<3-> \textbf{Eficientes}: operaciones eficientes gracias al uso de estructuras de datos y algoritmos
                    \item<4-> \textbf{Multi-usuarios}: protocolos para la gesti\'on de accesos concurrentes
                    \item<5-> \textbf{Seguros}: consistentes ante accesos por usuarios no autorizados y fallos del sistema
                    \item<6-> \textbf{Disponibles}: al 99.99999\%
                \end{itemize}
            \end{block}
               
\end{frame}


\begin{frame}{Superando expectativas}
    \begin{block}{Conveniencia de los SGBD}
        \begin{itemize}
            \item<2-> \textbf{Independencia f\'isica de datos}: admiten el cambio de la forma de
            almacenamiento de los datos, pero la estructura de la base de datos y las operaciones
            definidas sobre ella no cambian.
            \item<3-> \textbf{Independencia l\'ogica de datos}: proporcionan lenguajes de consulta declarativos, se define
            lo que se desea pero no c\'omo alcanzarlo.
        \end{itemize}
        
    \end{block}

\end{frame}


\begin{frame}{Superando las expectativas}
    \centering
    \begin{tikzpicture}
        \node at (1.4,7) {\tiny \bf {Usuario de SGBD}} ;
        \node[inner sep=0pt] at (1.4,5.8) {
            \includegraphics[width=1.8cm]{img/hface.png}
        };

        \node at (1.4,4.3) {\tiny \bf {Desarrollador de SGBD}} ;
        \node[inner sep=0pt] at (1.4,3.2) {
            \includegraphics[width=1.8cm]{img/sface.png}
        };

        \draw[-,thick] (0.3, 4.6) -- (7,4.5);
        \draw[-,thick] (2.7, 2) -- (2.7,7.4);

        \node at (5,5.8) {Lenguaje declarativo};
        \node at (5,4.3) {Estructuras de datos};%\\Algoritmos\\Optimizaci\'on\\Compilaci\'on\\Gesti\'on de ficheros};
        \node at (5,3.8) {Algoritmos};%\\Algoritmos\\Optimizaci\'on\\Compilaci\'on\\Gesti\'on de ficheros};
        \node at (5,3.3) {Optimizaci\'on};%\\Algoritmos\\Optimizaci\'on\\Compilaci\'on\\Gesti\'on de ficheros};
        \node at (5,2.8) {Compilaci\'on};%\\Algoritmos\\Optimizaci\'on\\Compilaci\'on\\Gesti\'on de ficheros};
        \node at (5,2.3) {Gesti\'on de ficheros};%\\Algoritmos\\Optimizaci\'on\\Compilaci\'on\\Gesti\'on de ficheros};

    \end{tikzpicture}
\end{frame}
\begin{frame}{Sistemas de bases de datos (SBD)}
  
    \centering
    \begin{tikzpicture}<1->
        \node at (0,3.7) {\tiny \bf {Usuario}} ;
        \node[inner sep=0pt] (user) at (0,3) {
            \includegraphics[width=1.3cm]{img/user.png}
        };

        \node at (3,3.7) {\tiny \bf {Aplicaci\'on}} ;
        \node[inner sep=0pt] (desktop) at (3,3) {
            \includegraphics[width=1.3cm]{img/desktop.png}
        };

        \node at (6,3.7) {\tiny \bf {SGBD}} ;
        \node[inner sep=0pt] (dbms) at (6,3) {
            \includegraphics[width=1.3cm]{img/dbms.png}
        };

        \node at (9,3.7) {\tiny \bf {Base de datos}} ;
        \node[inner sep=0pt] (database) at (9,3) {
            \includegraphics[width=1.3cm]{img/database.png}
        };

        \draw[<->,thick] (user.east) -- (desktop.west);
        \draw[<->,thick] (desktop.east) -- (dbms.west);
        \draw[<->,thick] (dbms.east) -- (database.west);
    \end{tikzpicture}
    \vspace{10pt}
    \begin{block}<2->{}
        \begin{quote}
            ``Un \textcolor{red}{sistema de base de datos} es un 
            \textcolor{red}{sistema computacional de mantenimiento de registros}, 
            que se dise\~na para manejar grandes cantidades de informaci\'on."
            \hspace{1em plus 1fill}---C. J. Date
        \end{quote}
       
    \end{block}

    \begin{block}<3->{Funciones}
        \begin{columns}[T]
            \begin{column}{0.48\linewidth}
                \begin{itemize}
                    \item Insertar datos
                    \item Editar datos
                \end{itemize}
            \end{column}
            \begin{column}{0.48\linewidth}
                \begin{itemize}
                    \item Eliminar datos
                    \item Consultar datos
                \end{itemize}
            \end{column}
            
        \end{columns}
    \end{block}
  
\end{frame}



\begin{frame}{Evoluci\'on de las tecnolog\'ias de la informaci\'on}

    \begin{overlayarea}{\linewidth}{\textheight}
        \vspace{10mm}
        \centering
    \begin{tikzpicture}[xscale=0.15]

        \node[rectangle, fill=blue, minimum width=14mm] at (0,0) {};
        \node[anchor=south, color=blue, inner sep=0pt] at (0.2,0.3) {\small 1980};

        \node[rectangle, fill=black!20, minimum width=14mm] at (10,0) {};
        \node[anchor=south, color=black!20, inner sep=0pt] at (10.2,0.3) {\small 1990};

        \node[rectangle, fill=black!20, minimum width=14mm] at (20,0) {};
        \node[anchor=south, color=black!20, inner sep=0pt] at (20.2,0.3) {\small 2000};

        \node[rectangle, fill=black!20, minimum width=14mm] at (30,0) {};
        \node[anchor=south, color=black!20, inner sep=0pt] at (30.2,0.3) {\small 2010};
    \end{tikzpicture}

    \vspace{20pt}

    \begin{columns}[T]
        \begin{column}{0.23\textwidth}
            \centering
            \includegraphics[width=1.3cm]{img/sqldb_nb.png}\\
            Primeros sistemas gestores de bases de datos relacionales comerciales
        \end{column}
        \begin{column}{0.23\textwidth}
            \centering
            \includegraphics[width=1.3cm]{img/sql.png}\\
            Est\'andar de lenguaje SQL
        \end{column}

        \begin{column}{0.23\textwidth}
            \centering
            \includegraphics[width=1.3cm]{img/server.png}\\
            {Primeros servidores de bases de datos relacionales}
        \end{column}
        \begin{column}{0.23\textwidth}
            \centering
            \includegraphics[width=1.3cm]{img/monitor.png}\\
            Procesamiento anal\'itico de datos
        \end{column}
    \end{columns}
    \end{overlayarea}
   

\end{frame}



\begin{frame}{Evoluci\'on de las tecnolog\'ias de la informaci\'on}
    \begin{overlayarea}{\textwidth}{\textheight}
        \vspace{10mm}
    \centering
    \begin{tikzpicture}[xscale=0.15]

        \node[rectangle, fill=black!20, minimum width=14mm] at (0,0) {};
        \node[anchor=south, color=black!20, inner sep=0pt] at (0.2,0.3) {\small 1980};

        \node[rectangle, fill=blue, minimum width=14mm] at (10,0) {};
        \node[anchor=south, color=blue, inner sep=0pt] at (10.2,0.3) {\small 1990};

        \node[rectangle, fill=black!20, minimum width=14mm] at (20,0) {};
        \node[anchor=south, color=black!20, inner sep=0pt] at (20.2,0.3) {\small 2000};

        \node[rectangle, fill=black!20, minimum width=14mm] at (30,0) {};
        \node[anchor=south, color=black!20, inner sep=0pt] at (30.2,0.3) {\small 2010};
    \end{tikzpicture}

    \vspace{20pt}

    \begin{columns}[T]
        \begin{column}{0.23\textwidth}
            \centering
            \includegraphics[width=1.3cm]{img/website.png}\\
            Internet
        \end{column}
        \begin{column}{0.23\textwidth}
            \centering
            \includegraphics[width=1.3cm]{img/online-store.png}\\
            Negocios online
        \end{column}

        \begin{column}{0.23\textwidth}
            \centering
            \includegraphics[width=1.3cm]{img/xml.png}\\
            Modelo orientado a objetos\\ XML
        \end{column}
    \end{columns}
\end{overlayarea}

\end{frame}


\begin{frame}{Evoluci\'on de las tecnolog\'ias de la informaci\'on}
    \begin{overlayarea}{\textwidth}{\textheight}
        \vspace{10mm}
    \centering
    \begin{tikzpicture}[xscale=0.15]

        \node[rectangle, fill=black!20, minimum width=14mm] at (0,0) {};
        \node[anchor=south, color=black!20, inner sep=0pt] at (0.2,0.3) {\small 1980};

        \node[rectangle, fill=black!20, minimum width=14mm] at (10,0) {};
        \node[anchor=south, color=black!20, inner sep=0pt] at (10.2,0.3) {\small 1990};

        \node[rectangle, fill=blue, minimum width=14mm] at (20,0) {};
        \node[anchor=south, color=blue, inner sep=0pt] at (20.2,0.3) {\small 2000};

        \node[rectangle, fill=black!20, minimum width=14mm] at (30,0) {};
        \node[anchor=south, color=black!20, inner sep=0pt] at (30.2,0.3) {\small 2010};
    \end{tikzpicture}

    \vspace{20pt}

    \begin{columns}[T]
        \begin{column}{0.23\textwidth}
            \centering
            \includegraphics[width=1.3cm]{img/warehouse.png}\\
            Data Warehousing
        \end{column}

        \begin{column}{0.23\textwidth}
            \centering
            \includegraphics[width=1.3cm]{img/networking.png}\\
            Redes sociales
        \end{column}
        \begin{column}{0.23\textwidth}
            \centering
            \includegraphics[width=1.3cm]{img/booking.png}\\
            Computaci\'on m\'ovil
        \end{column}

        \begin{column}{0.23\textwidth}
            \centering
            \includegraphics[width=1.3cm]{img/big-data.png}\\
            Inicios del Big Data
        \end{column}

    \end{columns}
\end{overlayarea}

\end{frame}



\begin{frame}{Evoluci\'on de las tecnolog\'ias de la informaci\'on}
    \begin{overlayarea}{\textwidth}{\textheight}
        \vspace{10mm}
    \centering
    \begin{tikzpicture}[xscale=0.15]

        \node[rectangle, fill=black!20, minimum width=14mm] at (0,0) {};
        \node[anchor=south, color=black!20, inner sep=0pt] at (0.2,0.3) {\small 1980};

        \node[rectangle, fill=black!20, minimum width=14mm] at (10,0) {};
        \node[anchor=south, color=black!20, inner sep=0pt] at (10.2,0.3) {\small 1990};

        \node[rectangle, fill=black!20, minimum width=14mm] at (20,0) {};
        \node[anchor=south, color=black!20, inner sep=0pt] at (20.2,0.3) {\small 2000};

        \node[rectangle, fill=blue, minimum width=14mm] at (30,0) {};
        \node[anchor=south, color=blue, inner sep=0pt] at (30.2,0.3) {\small 2010};
    \end{tikzpicture}

    \vspace{20pt}

    \begin{columns}[T]
        
        \begin{column}{0.23\textwidth}
            \centering
            \includegraphics[width=1.6cm]{img/noSQL.png}\\
            Sistemas gestores de bases de datos NoSQL
        \end{column}

        \begin{column}{0.23\textwidth}
            \centering
            \includegraphics[width=1.3cm]{img/cloud-computing.png}\\
            Computaci\'on en la nube
        \end{column}
    \end{columns}
\end{overlayarea}

\end{frame}
\begin{frame}
\begin{textblock*}{\paperwidth}(0mm,0mm) % {block width} (coords)
    \includegraphics[width=\paperwidth,height=\paperheight]{img/hierarchy.png}
  \end{textblock*}
\end{frame}
\begin{frame}{Objetivos de la asignatura}
    % \begin{block}{Bases de datos relacionales}
    %     \begin{itemize}
    %         \item<1-> Orientadas a almacenar datos estructurados (datos tabulares)
    %         \item<2-> Basadas en el modelo relacional (modelo matem\'atico de datos)
    %         \item<3-> Permiten el procesamiento transaccional de datos
    %         \item<4-> Base de los sistemas de informaci\'on para la generaci\'on de conocimiento
    %     \end{itemize}
    % \end{block}

    \begin{block}{Proporcionar un conjunto de m\'etodos y herramientas para:}
        \pause
        \begin{itemize}[<+->]
            \item Dise\~nar e implementar bases de datos correctas
            \item Evaluar la calidad de bases de datos espec\'ificas
            \item Identificar la vigencia del modelo relacional y sus limitaciones
            \item Reconocer casos de uso para bases de datos no relacionales
        \end{itemize}        
    \end{block}

    \note<5>{@NOTE reformular en t\'erminos de insistir en la parte anal\'itica y en la diversidad de enfoques nosql}
\end{frame}
\begin{frame}{Estructura de la asignatura}
    % \begin{columns}[T]
    %     \begin{column}{0.24\linewidth}
    %         \begin{tikzpicture}
    %             \node[rectangle, fill=violet, minimum width=\linewidth] at (0,0) {};
    %         \end{tikzpicture}
    %         \centering
    %         An\'alisis de Requerimientos
    %     \end{column}
    %     \begin{column}{0.24\linewidth}
    %         \begin{tikzpicture}
    %             \node[rectangle, fill=cyan, minimum width=\linewidth] at (0,0) {};
    %         \end{tikzpicture}
    %         \centering
    %         Dise\~no Conceptual
    %     \end{column}
    %     \begin{column}{0.24\linewidth}
    %         \begin{tikzpicture}
    %             \node[rectangle, fill=green, minimum width=\linewidth] at (0,0) {};
    %         \end{tikzpicture}
    %         \centering
    %         Dise\~no\\ L\'ogico
    %     \end{column}
    %     \begin{column}{0.24\linewidth}
    %         \begin{tikzpicture}
    %             \node[rectangle, fill=blue, minimum width=\linewidth] at (0,0) {};
    %         \end{tikzpicture}
    %         \centering
    %         Dise\~no\\ F\'isico
    %     \end{column}
    \begin{overlayarea}{\linewidth}{\textheight}
        \vspace{5mm}
        \centering
        \begin{tikzpicture}
            \node<1->[rectangle, fill=blue1, minimum width=5cm, minimum height=8mm, text=white] at (0,0) {An\'alisis de Requerimientos};
            \node<2->[single arrow, fill=black, inner sep=6pt, rotate=270] at (0,-0.8){};
            \node<2->[rectangle, fill=blue2, minimum width=5cm, minimum height=8mm, text=white] at (0,-1.8) {Dise\~no Conceptual};
            \node<3->[single arrow, fill=black, inner sep=6pt, rotate=270] at (0,-2.6){};
            \node<3->[rectangle, fill=blue3, minimum width=5cm, minimum height=8mm, text=white] at (0,-3.6) {Dise\~no L\'ogico};
            \node<4->[single arrow, fill=black, inner sep=6pt, rotate=270] at (0,-4.4){};
            \node<4->[rectangle, fill=blue4, minimum width=5cm, minimum height=8mm, text=white] at (0,-5.4) {Dise\~no F\'isico};
        \end{tikzpicture}
            
    \end{overlayarea}
        
\end{frame}
\begin{frame}{Evaluaci\'on de la asignatura}
    \begin{columns}[T]
        \begin{column}{0.40\linewidth}
            \begin{block}{Evaluaci\'on Sistem\'atica}
        
                \begin{itemize}
                    \item Trabajos de Control Parcial
                    \item<2-> Participaci\'on en clases
                    \item<2-> Entrega de tareas orientadas
                    \item<2-> Asistencia
                \end{itemize}
            \end{block}    
            \vspace{10pt}
            \begin{block}<3->{Nota final}
                
                \begin{itemize}
                    \item Prueba Final
                    \item Opini\'on del profesor
                \end{itemize}
            \end{block}
        \end{column}
        
        \begin{column}{0.58\linewidth}
            \includegraphics[width=0.8\textwidth, height=0.8\textheight]{img/eval.jpg}
        \end{column}
    \end{columns}

   
\end{frame}
\begin{frame}{Recursos}
   
            \begin{block}{Bibliograf\'ia}
                \nocite{sdb-book}
                \nocite{sql}
                \printbibliography
            \end{block}

            \begin{block}{\textit{Software}}
                \begin{itemize}
                    \item MySQL 8.0
                    \item Python
                    \item Jupyter Notebook
                \end{itemize}
            \end{block}
            
            \centering
            \includegraphics[width=1cm]{img/telegram.png}
            
            @matcom\_databases\_cs
     
      
           
\end{frame}
\maketitle
\end{document}