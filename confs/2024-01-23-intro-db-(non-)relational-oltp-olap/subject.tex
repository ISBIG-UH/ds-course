\begin{frame}{Objetivos de la asignatura}
    % \begin{block}{Bases de datos relacionales}
    %     \begin{itemize}
    %         \item<1-> Orientadas a almacenar datos estructurados (datos tabulares)
    %         \item<2-> Basadas en el modelo relacional (modelo matem\'atico de datos)
    %         \item<3-> Permiten el procesamiento transaccional de datos
    %         \item<4-> Base de los sistemas de informaci\'on para la generaci\'on de conocimiento
    %     \end{itemize}
    % \end{block}

    \begin{block}{Proporcionar un conjunto de m\'etodos y herramientas para:}
        \pause
        \begin{itemize}[<+->]
            \item Dise\~nar e implementar bases de datos correctas
            \item Evaluar la calidad de bases de datos espec\'ificas
            \item Identificar la vigencia del modelo relacional y sus limitaciones
            \item Reconocer casos de uso para bases de datos no relacionales
        \end{itemize}        
    \end{block}

    \note<5>{@NOTE reformular en t\'erminos de insistir en la parte anal\'itica y en la diversidad de enfoques nosql}
\end{frame}