\begin{frame}{Bases de datos}
    \begin{overlayarea}{\linewidth}{\textheight}
            \begin{block}{}
                \begin{quote}
                    ``... una base de datos es una colecci\'on \textcolor<3>{blue}{auto-descriptiva} de \textcolor<2>{blue}{registros} \textcolor<4>{blue}{integrados}."
                    \hspace{1em plus 1fill}---Allen Taylor
                \end{quote}
            \end{block}
                
    \begin{block}<2->{Registro}
        Datos espec\'ificos sobre una entidad u objeto de inter\'es.
    \end{block}

    \begin{block}<3->{Auto-descriptiva}
        Se almacenan metadatos (la descripci\'on de su estructura) dentro.
    \end{block}

    \begin{block}<4->{Integrados}
        No solo contiene los datos sino tambi\'en las interrelaciones.
    \end{block}
    \end{overlayarea}

    \pause
   
   \note<5>{@NOTE Ahora vamos a ir explicando c\'omo se ajustan los t\'erminos \textit{auto-descriptiva}, \textit{registros} e \textit{integrados} a cada tipo de BDs que vamos a estudiar}
\end{frame}

\begin{frame}
    \frametitle{Bases de datos}
    \framesubtitle{Relacionales}

\begin{center}
    \Huge \textbf{Relacionales}
\end{center}
\end{frame}

\begin{frame}{Bases de datos}
    \framesubtitle{Relacionales}

    \begin{overlayarea}{\linewidth}{\textheight}
        \begin{onlyenv}
            \begin{block}{}
                \begin{quote}
                    ``... una base de datos es una colecci\'on auto-descriptiva de \textcolor{red}{registros} integrados."
                    \hspace{1em plus 1fill}---Allen Taylor
                \end{quote}
                
                \textcolor{red}{Registro}: datos espec\'ificos sobre una entidad u objeto de inter\'es
            \end{block}
      \end{onlyenv}
      \vspace{8mm}
      \centering
      \begin{tabular}{|c|c|c|c|c|}
          \hline
          23082300205 & Edgar F. & Codd & 23/08/1923\\
          \hline
      \end{tabular}
    \end{overlayarea}
\end{frame}

\begin{frame}{Bases de datos}
    \framesubtitle{Relacionales}

    \begin{overlayarea}{\linewidth}{\textheight}
        \begin{onlyenv}
            \begin{block}{}
            \begin{quote}
                ``... una base de datos es una colecci\'on \textcolor{red}{auto-descriptiva} de registros integrados."
                \hspace{1em plus 1fill}---Allen Taylor
            \end{quote}
    
            \textcolor{red}{Auto-descriptiva}: se almacenan metadatos (la descripci\'on de su estructura) dentro
            del diccionario de datos de la propia base de datos.
        \end{block}
      \end{onlyenv}
      
          \vspace{5mm}
          \large \textbf{Persona}
          \vspace{2mm}
          \centering{
          
      
          \begin{tabular}{|c|c|c|c|}
              \hline
              CI : \textcolor{blue}{\texttt{string}} & Nombre : \textcolor{blue}{\texttt{string}} & Apellido : \textcolor{blue}{\texttt{string}} & F. Nacimiento : \textcolor{blue}{\texttt{date}} \\ 
              \hline
            23082300205 & Edgar F. & Codd & 23/08/1923\\
              \hline
          \end{tabular}
          }
    \end{overlayarea}
\end{frame}



\begin{frame}{Bases de datos}
    \framesubtitle{Relacionales}

    \begin{overlayarea}{\linewidth}{\textheight}
        \begin{onlyenv}
            \begin{block}{}
                \begin{quote}
                    ``... una base de datos es una colecci\'on auto-descriptiva de registros \textcolor{red}{integrados}."
                    \hspace{1em plus 1fill}---Allen Taylor
                \end{quote}
                \textcolor{red}{Integrados}: no solo contiene los datos sino tambi\'en las interrelaciones
                 que se establecen entre estos.
            \end{block}
      \end{onlyenv}

      \vspace{5mm}
    \large \textbf{Persona}
    \vspace{2mm}
    \centering{
    

    \begin{tabular}{|c|c|c|c|}
        \hline
        CI : \textcolor{blue}{\texttt{string}} & Nombre : \textcolor{blue}{\texttt{string}} & Apellido : \textcolor{blue}{\texttt{string}} & F. Nacimiento : \textcolor{blue}{\texttt{date}} \\ 
        \hline
        \textcolor{red}{23082300205} & Edgar F. & Codd & 23/08/1923\\
        \hline
    \end{tabular}
    }

    \vspace{5mm}


    \large \textbf{Cuenta}

    \vspace{2mm}
    \centering{
    \begin{tabular}{|c|c|c|}
        \hline
        No. Cuenta : \textcolor{blue}{\texttt{string}} & Balance : \textcolor{blue}{\texttt{decimal}} & CI Due\~no : \textcolor{blue}{\texttt{string}} \\
        \hline
        8976 & 270.98 & \textcolor{red}{23082300205} \\
        \hline
    \end{tabular}
    }
    \end{overlayarea}
    

    \note{@NOTE todo esto permite garantizar la consistencia en todo momento, lo cual es la principal fortaleza de las bds relacionales}
\end{frame}


\begin{frame}
    \frametitle{Bases de datos}
    \framesubtitle{Llave-valor}

\begin{center}
    \Huge \textbf{Llave-valor}
\end{center}
\end{frame}

\begin{frame}[fragile]{Bases de datos}
    \framesubtitle{Llave-valor}

    \begin{overlayarea}{\linewidth}{\textheight}
        \begin{onlyenv}
            \begin{block}{}
                \begin{quote}
                    ``... una base de datos es una colecci\'on auto-descriptiva de \textcolor{red}{registros} integrados."
                    \hspace{1em plus 1fill}---Allen Taylor
                \end{quote}
                
                \textcolor{red}{Registro}: datos espec\'ificos sobre una entidad u objeto de inter\'es
            \end{block}
      \end{onlyenv}
      \vspace{2mm}
      \begin{center}
        
\begin{minipage}{0.6\textwidth}
    \begin{onlyenv}<1>
\begin{lstlisting}[language=text]
`persona_23082300205': `{  
    "CI": 23082300205, 
    "Nombre": "Edgar F.", 
    "Apellido": "Codd", 
    "F. Nacimiento": 23/08/1923 
}'
\end{lstlisting}
    \end{onlyenv}

    \begin{onlyenv}<2>
\begin{lstlisting}[language=text]
`#persona_23082300205#': `{  
    "CI": 23082300205, 
    "Nombre": "Edgar F.", 
    "Apellido": "Codd", 
    "F. Nacimiento": 23/08/1923 
}'
\end{lstlisting}
    \end{onlyenv}

\begin{onlyenv}<3>
\begin{lstlisting}[language=text]
`persona_23082300205': `#{  
    "CI": 23082300205, 
    "Nombre": "Edgar F.", 
    "Apellido": "Codd", 
    "F. Nacimiento": 23/08/1923 
}#'
\end{lstlisting}
\end{onlyenv}

\end{minipage}
      \end{center}
    \end{overlayarea}

    \note<1>{@NOTE el objetivo de estas bds es garantizar el acceso eficiente al valor asociado a una llave. Vamos a ver m\'as adelante c\'omo esto se logra}
    \note<2>{@NOTE la llave}
    \note<3>{@NOTE el valor \textbf{no se interpreta, no se asume estructurado, se almacena tal cual}. Se espera un \textit{blob} o string}
\end{frame}

\begin{frame}[fragile]{Bases de datos}
    \framesubtitle{Llave-valor}

    \begin{overlayarea}{\linewidth}{\textheight}
        \begin{onlyenv}
            \begin{block}{}
            \begin{quote}
                ``... una base de datos es una colecci\'on \textcolor{red}{auto-descriptiva} de registros integrados."
                \hspace{1em plus 1fill}---Allen Taylor
            \end{quote}
    
            \textcolor{red}{Auto-descriptiva}: se almacenan metadatos (la descripci\'on de su estructura) dentro
            del diccionario de datos de la propia base de datos.
        \end{block}
      \end{onlyenv}
      
          \vspace{5mm}
\centering

\begin{minipage}{0.6\textwidth}
\begin{lstlisting}[language=text]
`persona_23082300205': `{  
    "#CI#": 23082300205, 
    "#Nombre#": "Edgar F.", 
    "#Apellido#": "Codd", 
    "#F. Nacimiento#": 23/08/1923 
}'
\end{lstlisting}

\end{minipage}
    \end{overlayarea}

                \note{@NOTE no hay esquema. Mayor libertad. La integridad debe ser garantizada a nivel de app}
\end{frame}



\begin{frame}[fragile]{Bases de datos}
    \framesubtitle{Llave-valor}

    \begin{overlayarea}{\linewidth}{\textheight}
        \begin{onlyenv}
            \begin{block}{}
                \begin{quote}
                    ``... una base de datos es una colecci\'on auto-descriptiva de registros \textcolor{red}{integrados}."
                    \hspace{1em plus 1fill}---Allen Taylor
                \end{quote}
                \textcolor{red}{Integrados}: no solo contiene los datos sino tambi\'en las interrelaciones
                 que se establecen entre estos.
            \end{block}
      \end{onlyenv}

      \vspace{5mm}
        \begin{columns}
            \begin{column}[t]{0.5\textwidth}
                \centering

                \begin{lstlisting}[language=text]
`persona_#23082300205#': `{  
    "CI": 23082300205, 
    "Nombre": "Edgar F.", 
    "Apellido": "Codd", 
    "F. Nacimiento": 23/08/1923 
}'
                \end{lstlisting} 
            \end{column}

            \begin{column}[t]{0.5\textwidth}
                \centering

                \begin{lstlisting}[language=text]
`cuenta_8976': `{  
    "No. Cuenta": 8976, 
    "Balance": 270.98, 
    "CI Duenyo": #23082300205#
}'
                \end{lstlisting} 
            \end{column}
        \end{columns}
    \end{overlayarea}
    
    \note{@NOTE mantener la consistencia e integridad va a nivel de app. No es el objetivo principal de las BDs llave-valor}
\end{frame}

\begin{frame}
    \frametitle{Bases de datos}
    \framesubtitle{Columnares}

\begin{center}
    \Huge \textbf{Columnares}
\end{center}
\end{frame}

\begin{frame}{Bases de datos}
    \framesubtitle{Columnares}

    \begin{overlayarea}{\linewidth}{\textheight}
            \begin{block}{}
                \begin{quote}
                    ``... una base de datos es una colecci\'on auto-descriptiva de \textcolor{red}{registros} integrados."
                    \hspace{1em plus 1fill}---Allen Taylor
                \end{quote}
                
                \textcolor{red}{Registro}: datos espec\'ificos sobre una entidad u objeto de inter\'es
            \end{block}
      \vspace{8mm}

      \centering

        \begin{onlyenv}<1>
      \begin{tabular}{|c|}
          \hline
           23082300205 \\ \hline
           Edgar F. \\\hline
           Codd \\\hline
           23/08/1923 \\
          \hline
      \end{tabular}
    \end{onlyenv}
     
     \begin{onlyenv}<2->
      \begin{tabular}{|c|c|c|}
          \hline
           23082300205 & 43101123919 & 44011279165 \\ \hline
           Edgar F. & Michael & Jim \\\hline
           Codd & Stonebreaker & Gray \\\hline
           23/08/1923 & 11/10/1943 & 12/01/1944 \\
          \hline
      \end{tabular}
     \end{onlyenv}

    \end{overlayarea}

    \note<1>{@NOTE ahora el enfoque est\'a sobre las columnas, en lugar de las filas, i.e. es m\'as importante extraer todos los valores de una columna que todos los de una fila. Aqu\'i el principal objetivo es analizar comportamientos a partir de los datos}    
\end{frame}

\begin{frame}{Bases de datos}
    \framesubtitle{Columnares}

    \begin{overlayarea}{\linewidth}{\textheight}
        \begin{onlyenv}
            \begin{block}{}
            \begin{quote}
                ``... una base de datos es una colecci\'on \textcolor{red}{auto-descriptiva} de registros integrados."
                \hspace{1em plus 1fill}---Allen Taylor
            \end{quote}
    
            \textcolor{red}{Auto-descriptiva}: se almacenan metadatos (la descripci\'on de su estructura) dentro
            del diccionario de datos de la propia base de datos.
        \end{block}
      \end{onlyenv}
      
          \vspace{5mm}
      
        \begin{columns}
            \begin{column}[t]{0.5\textwidth}
          \centering

          \large \textbf{Persona}
          \vspace{2mm}

      \begin{tabular}{|c|c|}
          \hline
          CI : \textcolor{blue}{string} & 23082300205 \\ \hline
          Nombre : \textcolor{blue}{string} & Edgar F. \\\hline
          Apellido : \textcolor{blue}{string} & Codd \\\hline
          F. Nacimiento : \textcolor{blue}{date} & 23/08/1923 \\
          \hline
      \end{tabular}
    \end{column}
            \begin{column}[t]{0.5\textwidth}
            \end{column}
        \end{columns}

    \end{overlayarea}
\end{frame}



\begin{frame}{Bases de datos}
    \framesubtitle{Columnares}

    \begin{overlayarea}{\linewidth}{\textheight}
        \begin{onlyenv}
            \begin{block}{}
                \begin{quote}
                    ``... una base de datos es una colecci\'on auto-descriptiva de registros \textcolor{red}{integrados}."
                    \hspace{1em plus 1fill}---Allen Taylor
                \end{quote}
                \textcolor{red}{Integrados}: no solo contiene los datos sino tambi\'en las interrelaciones
                 que se establecen entre estos.
            \end{block}
      \end{onlyenv}

      \vspace{5mm}
        \begin{columns}
            \begin{column}[t]{0.5\textwidth}
          \centering

          \large \textbf{Persona}
          \vspace{2mm}

      \begin{tabular}{|c|c|}
          \hline
          CI : \textcolor{blue}{string} & \textcolor{red}{23082300205} \\ \hline
          Nombre : \textcolor{blue}{string} & Edgar F. \\\hline
          Apellido : \textcolor{blue}{string} & Codd \\\hline
          F. Nacimiento : \textcolor{blue}{date} & 23/08/1923 \\
          \hline
      \end{tabular}
    \end{column}

            \begin{column}[t]{0.5\textwidth}
          \centering

          \large \textbf{Cuenta}
          \vspace{2mm}

                \begin{tabular}{|c|c|}
                    \hline
                    No. Cuenta : \textcolor{blue}{string} & 8976 \\\hline
                    Balance : \textcolor{blue}{decimal} & 270.98 \\\hline
                    CI Due\~no : \textcolor{blue}{string} & \textcolor{red}{23082300205} \\
                    \hline
                \end{tabular}
            \end{column}
        \end{columns}
    \end{overlayarea}
    

    
\end{frame}

\begin{frame}
    \frametitle{Bases de datos}
    \framesubtitle{de Documentos}

\begin{center}
    \Huge \textbf{de Documentos}
\end{center}
\end{frame}

\begin{frame}[fragile]{Bases de datos}
    \framesubtitle{de Documentos}

    \begin{overlayarea}{\linewidth}{\textheight}
        \begin{onlyenv}
            \begin{block}{}
                \begin{quote}
                    ``... una base de datos es una colecci\'on \textcolor<2>{red}{auto-descriptiva} de \textcolor<1>{red}{registros} integrados."
                    \hspace{1em plus 1fill}---Allen Taylor
                \end{quote}
            \end{block}
      \end{onlyenv}

      \vspace{5mm}

      \begin{center}
        
        \begin{minipage}{.6\textwidth}
            \begin{onlyenv}<1>
        \begin{lstlisting}[language=json]
{  
    "CI": "23082300205", 
    "Nombre": "Edgar F.", 
    "Apellido": "Codd", 
    "F. Nacimiento": "23/08/1923",
    "Titulo de Doctor": {
        "F. Expedido": 1965,
        ...
    }
}
        \end{lstlisting}
            \end{onlyenv}

\begin{onlyenv}<2>
        \begin{lstlisting}[language=json]
{  
    "#CI#": "23082300205", 
    "#Nombre#": "Edgar F.", 
    "#Apellido#": "Codd", 
    "#F. Nacimiento#": "23/08/1923",
    "#Titulo de Doctor#": {
        "#F. Expedido#": 1965,
        ...
    }
}
        \end{lstlisting}
\end{onlyenv}
        \end{minipage}
      \end{center}
    \end{overlayarea}

    \note<1>{@NOTE han trabajado con json antes?}
    \note<2>{@NOTE el esquema puede variar entre docs de la misma colecci\'on, pero la estructura interna de un doc se encuentra bien definida}
\end{frame}

\begin{frame}[fragile]{Bases de datos}
    \framesubtitle{de Documentos}

    \begin{overlayarea}{\linewidth}{\textheight}
        \begin{onlyenv}
            \begin{block}{}
                \begin{quote}
                    ``... una base de datos es una colecci\'on auto-descriptiva de registros \textcolor{red}{integrados}."
                    \hspace{1em plus 1fill}---Allen Taylor
                \end{quote}
                \textcolor{red}{Integrados}: no solo contiene los datos sino tambi\'en las interrelaciones
                 que se establecen entre estos.
            \end{block}
      \end{onlyenv}

      \vspace{3mm}
        \begin{columns}
            \begin{column}[t]{0.525\textwidth}
                \centering
                \large \textbf{Persona} (llave:  ``\textcolor{blue}{23082300205}'')

        \begin{lstlisting}[language=json]
{  
    "CI": "23082300205", 
    "Nombre": "Edgar F.", 
    "Apellido": "Codd", 
    "F. Nacimiento": "23/08/1923"
}
        \end{lstlisting}
            \end{column}

            \begin{column}[t]{0.475\textwidth}
                \centering
                \large \textbf{Cuenta} (llave: ``8976'')

        \begin{lstlisting}[language=json]
{  
    "No. Cuenta": "8976", 
    "Balance": 270.98, 
    "CI Duenyo": "#23082300205#"
}
                \end{lstlisting} 
            \end{column}
        \end{columns}
    \end{overlayarea}
    
    \note{@NOTE dentro de un doc puede haber otro doc. Esta es otra manera de expresar una interrelaci\'on entre ellos}
\end{frame}

\begin{frame}
    \frametitle{Bases de datos}
    \framesubtitle{de Grafos}

\begin{center}
    \Huge \textbf{de Grafos}
\end{center}
\end{frame}

\begin{frame}
    \frametitle{?`Qu\'e es un grafo?}

    \begin{center}
        \includegraphics<2>[scale=.9]{img/graph-db/graph-vertices.pdf}
        \includegraphics<3>[scale=.9]{img/graph-db/graph-edges.pdf}
        \includegraphics<4>[scale=.9]{img/graph-db/graph-directed-edges.pdf}
    \end{center}

    \note<4>{@NOTE di casos de uso}
\end{frame}

\begin{frame}[fragile]{Bases de datos}
    \framesubtitle{de Grafos}

    \begin{overlayarea}{\linewidth}{\textheight}
        \begin{onlyenv}
            \begin{block}{}
                \begin{quote}
                    ``... una base de datos es una colecci\'on auto-descriptiva de \textcolor{red}{registros} integrados."
                    \hspace{1em plus 1fill}---Allen Taylor
                \end{quote}
                
                \textcolor{red}{Registro}: datos espec\'ificos sobre una entidad u objeto de inter\'es
            \end{block}
      \end{onlyenv}
      \vspace{2mm}
      \begin{center}
        
        % @TODO las etiketas de las imgs se ven fula en el edge web browser
        \includegraphics[scale=.77]{img/graph-db/graph-db-data.pdf}

      \end{center}
    \end{overlayarea}

    \note{@NOTE las interrelaciones tambi\'en pueden tener datos asociados}
\end{frame}

\begin{frame}[fragile]{Bases de datos}
    \framesubtitle{de Grafos}

    \begin{overlayarea}{\linewidth}{\textheight}
        \begin{onlyenv}
            \begin{block}{}
            \begin{quote}
                ``... una base de datos es una colecci\'on \textcolor{red}{auto-descriptiva} de registros integrados."
                \hspace{1em plus 1fill}---Allen Taylor
            \end{quote}
    
            \textcolor{red}{Auto-descriptiva}: se almacenan metadatos (la descripci\'on de su estructura) dentro
            del diccionario de datos de la propia base de datos.
        \end{block}
      \end{onlyenv}
      
          \vspace{5mm}
\centering

      \includegraphics[scale=.7]{img/graph-db/graph-db-auto-descriptive.pdf}

    \end{overlayarea}

    \note{@NOTE no se fuerza un esquema, pero las etiquetas y las llaves de las propiedades se encargan de describir los datos}
\end{frame}



\begin{frame}[fragile]{Bases de datos}
    \framesubtitle{de Grafos}

    \begin{overlayarea}{\linewidth}{\textheight}
        \begin{onlyenv}
            \begin{block}{}
                \begin{quote}
                    ``... una base de datos es una colecci\'on auto-descriptiva de registros \textcolor{red}{integrados}."
                    \hspace{1em plus 1fill}---Allen Taylor
                \end{quote}
                \textcolor{red}{Integrados}: no solo contiene los datos sino tambi\'en las interrelaciones
                 que se establecen entre estos.
            \end{block}
      \end{onlyenv}

      \vspace{5mm}
      
      \includegraphics[scale=.74]{img/graph-db/graph-db-relations.pdf}
    \end{overlayarea}
    
    \note{@NOTE las interrelaciones son ciudadanas de 1ra clase}
\end{frame}
