\begin{frame}
    \frametitle{Bases de datos}
    \framesubtitle{de Documentos}

\begin{center}
    \Huge \textbf{de Documentos}
\end{center}
\end{frame}

\begin{frame}[fragile]{Bases de datos}
    \framesubtitle{de Documentos}

    \begin{overlayarea}{\linewidth}{\textheight}
        \begin{onlyenv}
            \begin{block}{}
                \begin{quote}
                    ``... una base de datos es una colecci\'on \textcolor<2>{red}{auto-descriptiva} de \textcolor<1>{red}{registros} integrados."
                    \hspace{1em plus 1fill}---Allen Taylor
                \end{quote}
            \end{block}
      \end{onlyenv}

      \vspace{5mm}

      \begin{center}
        
        \begin{minipage}{.6\textwidth}
            \begin{onlyenv}<1>
        \begin{lstlisting}[language=json]
{  
    "CI": "23082300205", 
    "Nombre": "Edgar F.", 
    "Apellido": "Codd", 
    "F. Nacimiento": "23/08/1923",
    "Titulo de Doctor": {
        "F. Expedido": 1965,
        ...
    }
}
        \end{lstlisting}
            \end{onlyenv}

\begin{onlyenv}<2>
        \begin{lstlisting}[language=json]
{  
    "#CI#": "23082300205", 
    "#Nombre#": "Edgar F.", 
    "#Apellido#": "Codd", 
    "#F. Nacimiento#": "23/08/1923",
    "#Titulo de Doctor#": {
        "#F. Expedido#": 1965,
        ...
    }
}
        \end{lstlisting}
\end{onlyenv}
        \end{minipage}
      \end{center}
    \end{overlayarea}

    \note<1>{@NOTE han trabajado con json antes?}
    \note<2>{@NOTE el esquema puede variar entre docs de la misma colecci\'on, pero la estructura interna de un doc se encuentra bien definida}
\end{frame}

\begin{frame}[fragile]{Bases de datos}
    \framesubtitle{de Documentos}

    \begin{overlayarea}{\linewidth}{\textheight}
        \begin{onlyenv}
            \begin{block}{}
                \begin{quote}
                    ``... una base de datos es una colecci\'on auto-descriptiva de registros \textcolor{red}{integrados}."
                    \hspace{1em plus 1fill}---Allen Taylor
                \end{quote}
                \textcolor{red}{Integrados}: no solo contiene los datos sino tambi\'en las interrelaciones
                 que se establecen entre estos.
            \end{block}
      \end{onlyenv}

      \vspace{3mm}
        \begin{columns}
            \begin{column}[t]{0.525\textwidth}
                \centering
                \large \textbf{Persona} (llave:  ``\textcolor{blue}{23082300205}'')

        \begin{lstlisting}[language=json]
{  
    "CI": "23082300205", 
    "Nombre": "Edgar F.", 
    "Apellido": "Codd", 
    "F. Nacimiento": "23/08/1923"
}
        \end{lstlisting}
            \end{column}

            \begin{column}[t]{0.475\textwidth}
                \centering
                \large \textbf{Cuenta} (llave: ``8976'')

        \begin{lstlisting}[language=json]
{  
    "No. Cuenta": "8976", 
    "Balance": 270.98, 
    "CI Duenyo": "#23082300205#"
}
                \end{lstlisting} 
            \end{column}
        \end{columns}
    \end{overlayarea}
    
    \note{@NOTE dentro de un doc puede haber otro doc. Esta es otra manera de expresar una interrelaci\'on entre ellos}
\end{frame}
