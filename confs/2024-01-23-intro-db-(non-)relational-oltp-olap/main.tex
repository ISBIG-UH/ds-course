\documentclass{beamer}

\usepackage[spanish]{babel}

\usetheme{Madrid}
\setbeamertemplate{footline}{} 
% \setbeamertemplate{navigation symbols}{}
\usecolortheme{default}
% \includeonlyframes{current}

\title{Introducción a las Bases de Datos}
\subtitle{Bases de Datos Relacionales y No Relacionales. OLTP y OLAP}
\author{Lic. Andy Ledesma Garc\'ia \\ 
Dra. C. Lucina Garc\'ia Hern\'andez}
\institute{Departamento de Computaci\'on\\
Facultad de Matem\'atica y Computaci\'on\\
Universidad de La Habana}
\date{22 de enero de 2024}

\begin{document}

\frame{\titlepage}

\section*{Contenido}

\begin{frame}% @TODO remove this
\frametitle{Parte 1: Evoluci\'on y Conceptos B\'asicos de las Bases de Datos}
\tableofcontents[part=1]  
\end{frame}


\part{Evoluci\'on y Conceptos B\'asicos de las Bases de Datos}

\section{Grandes Vol\'umenes de Informaci\'on}
\subsection{?`Qu\'e tan Grande es YouTube?}
\subsection{Hay muchas plataformas que manejan grandes vol\'umenes de info}
\subsection{Problemas a Resolver}

\section{Sistemas Orientados a Ficheros}
\subsection{Caracter\'isticas}
\subsection{Limitaciones}

\section{Sistemas de Bases de Datos}
\subsection{Definición de Bases de Datos}
\subsection{C\'omo cada tipo (no-relacionales incluidos) cumple la def}

% @TODO hablar de SBD (que' es distinto a SGBD)
% @TODO hacer e'nfasis en q nea4j, mongodb, etc son GESTORES DE BASES DE DATOS, NO BASES DE DATOS
\section{SGBD}
\subsection{Del enfoque manual al autom\'atico para interactuar con los datos}
\subsection{Definición de SGBD}
\subsection{?`C\'omo resuelven los problemas?}
\subsection{Ventajas que introducen}

% @TODO dentro de la evolucio'n introducir nosql y oltp vs olap

% @TODO presentar la asignatura

\frame{\partpage}

\end{document}