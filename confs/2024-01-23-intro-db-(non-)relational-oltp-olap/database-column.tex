\begin{frame}
    \frametitle{Bases de datos}
    \framesubtitle{Columnares}

\begin{center}
    \Huge \textbf{Columnares}
\end{center}
\end{frame}

\begin{frame}{Bases de datos}
    \framesubtitle{Columnares}

    \begin{overlayarea}{\linewidth}{\textheight}
            \begin{block}{}
                \begin{quote}
                    ``... una base de datos es una colecci\'on auto-descriptiva de \textcolor{red}{registros} integrados."
                    \hspace{1em plus 1fill}---Allen Taylor
                \end{quote}
                
                \textcolor{red}{Registro}: datos espec\'ificos sobre una entidad u objeto de inter\'es
            \end{block}
      \vspace{8mm}

      \centering

        \begin{onlyenv}<1>
      \begin{tabular}{|c|}
          \hline
           23082300205 \\ \hline
           Edgar F. \\\hline
           Codd \\\hline
           23/08/1923 \\
          \hline
      \end{tabular}
    \end{onlyenv}
     
     \begin{onlyenv}<2->
      \begin{tabular}{|c|c|c|}
          \hline
           23082300205 & 43101123919 & 44011279165 \\ \hline
           Edgar F. & Michael & Jim \\\hline
           Codd & Stonebreaker & Gray \\\hline
           23/08/1923 & 11/10/1943 & 12/01/1944 \\
          \hline
      \end{tabular}
     \end{onlyenv}

    \end{overlayarea}

    \note<1>{@NOTE ahora el enfoque est\'a sobre las columnas, en lugar de las filas, i.e. es m\'as importante extraer todos los valores de una columna que todos los de una fila. Aqu\'i el principal objetivo es analizar comportamientos a partir de los datos}    
\end{frame}

\begin{frame}{Bases de datos}
    \framesubtitle{Columnares}

    \begin{overlayarea}{\linewidth}{\textheight}
        \begin{onlyenv}
            \begin{block}{}
            \begin{quote}
                ``... una base de datos es una colecci\'on \textcolor{red}{auto-descriptiva} de registros integrados."
                \hspace{1em plus 1fill}---Allen Taylor
            \end{quote}
    
            \textcolor{red}{Auto-descriptiva}: se almacenan metadatos (la descripci\'on de su estructura) dentro
            del diccionario de datos de la propia base de datos.
        \end{block}
      \end{onlyenv}
      
          \vspace{5mm}
      
        \begin{columns}
            \begin{column}[t]{0.5\textwidth}
          \centering

          \large \textbf{Persona}
          \vspace{2mm}

      \begin{tabular}{|c|c|}
          \hline
          CI : \textcolor{blue}{string} & 23082300205 \\ \hline
          Nombre : \textcolor{blue}{string} & Edgar F. \\\hline
          Apellido : \textcolor{blue}{string} & Codd \\\hline
          F. Nacimiento : \textcolor{blue}{date} & 23/08/1923 \\
          \hline
      \end{tabular}
    \end{column}
            \begin{column}[t]{0.5\textwidth}
            \end{column}
        \end{columns}

    \end{overlayarea}
\end{frame}



\begin{frame}{Bases de datos}
    \framesubtitle{Columnares}

    \begin{overlayarea}{\linewidth}{\textheight}
        \begin{onlyenv}
            \begin{block}{}
                \begin{quote}
                    ``... una base de datos es una colecci\'on auto-descriptiva de registros \textcolor{red}{integrados}."
                    \hspace{1em plus 1fill}---Allen Taylor
                \end{quote}
                \textcolor{red}{Integrados}: no solo contiene los datos sino tambi\'en las interrelaciones
                 que se establecen entre estos.
            \end{block}
      \end{onlyenv}

      \vspace{5mm}
        \begin{columns}
            \begin{column}[t]{0.5\textwidth}
          \centering

          \large \textbf{Persona}
          \vspace{2mm}

      \begin{tabular}{|c|c|}
          \hline
          CI : \textcolor{blue}{string} & \textcolor{red}{23082300205} \\ \hline
          Nombre : \textcolor{blue}{string} & Edgar F. \\\hline
          Apellido : \textcolor{blue}{string} & Codd \\\hline
          F. Nacimiento : \textcolor{blue}{date} & 23/08/1923 \\
          \hline
      \end{tabular}
    \end{column}

            \begin{column}[t]{0.5\textwidth}
          \centering

          \large \textbf{Cuenta}
          \vspace{2mm}

                \begin{tabular}{|c|c|}
                    \hline
                    No. Cuenta : \textcolor{blue}{string} & 8976 \\\hline
                    Balance : \textcolor{blue}{decimal} & 270.98 \\\hline
                    CI Due\~no : \textcolor{blue}{string} & \textcolor{red}{23082300205} \\
                    \hline
                \end{tabular}
            \end{column}
        \end{columns}
    \end{overlayarea}
    

    
\end{frame}
