\begin{frame}
    \frametitle{Almacenamiento basado en columnas}

    \begin{block}{Principal caracter\'istica}
        Cada bloque de informaci\'on s\'olo contiene datos de una columna.
    \end{block}

    \begin{columns}<2->[t]
        \column{.5\textwidth}
\begin{exampleblock}{Ventajas}
    \begin{itemize}[<+(1)->]
        \item Requiere menos almacenamiento gracias a t\'ecnicas de compresi\'on
        \item Eficiente para consultas que requieren muchos datos pero s\'olo un peque\~no subconjunto de las columnas
        \item C\'omputo en paralelo
    \end{itemize}
\end{exampleblock}

        \column{.5\textwidth}
\begin{alertblock}{Desventajas}
    \begin{itemize}[<+(1)->]
        \item Las operaciones de escritura son costosas
        \item Ineficiente para operar sobre las filas
    \end{itemize}
\end{alertblock}
    \end{columns}

    \note<2>{@NOTE x ejemplo:
        \begin{itemize}
            \item referenciar a valores \'unicos
            \item eliminar redundancia en datos repetidos continuamente (Run-Length Encoding)
            \item eliminaci\'on de nulls llevando m\'ascaras de bits, por ejemplo
        \end{itemize}
    }
\end{frame}

\begin{frame}
    \frametitle{Almacenamiento basado en columnas}

    \begin{block}{Casos de uso}
        \begin{itemize}[<+->]
            \item \textit{Data warehousing} e Inteligencia de Negocios (BI)
            \item Series de tiempo
            \item An\'alisis en tiempo real
        \end{itemize}
    \end{block}
\end{frame}