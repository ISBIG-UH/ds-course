\begin{frame}{Continuemos con el ejemplo}
        $U = \{ \textnormal{\#E, ENombre,  Grupo, Provincia, \#A, ANombre, Nota}\}$

        $F = \{\textnormal{\#E} \to \textnormal{ENombre, Grupo, Provincia}${\bf;}\; $\textnormal{\#A} \to \textnormal{ANombre}${\bf;}\; $\textnormal{\#E,\#A} \to \textnormal{\#E,\#A}${\bf;}\; $\textnormal{\#E, \#A} \to \textnormal{Nota}${\bf;}\; $\textnormal{Provincia} \to \textnormal{Grupo}\}$\\[4mm]

    \centering
    \begin{tabular}{ccccccc}
        \underline{\#E} & ENombre & Grupo & Provincia & \underline{\#A} & ANombre & Nota\\
        $e_1$ & Juan & {111} & { La Habana} & $a_1$ & An\'alisis & 3\\
        $e_1$ & Juan & {111} & { La Habana} & $a_2$ & L\'ogica & 2\\
        $e_1$ & Juan & {111} & { La Habana} & $a_3$ & \'Algebra & 4\\
        $e_1$ & Juan & {111} & { La Habana} & $a_4$ & Programaci\'on & 5\\
        $e_3$ & Pedro & {111} & { La Habana} & $a_3$ & \'Algebra & 4\\
        $e_2$ & Mar\'ia & {112} & { Matanzas} & $a_1$ & An\'alisis & 3\\
        $e_2$ & Mar\'ia &  {112} & { Matanzas} & $a_2$ & L\'ogica & 3\\
        $e_4$ & Rita &  {112} & { Mayabeque} & $a_2$ & L\'ogica & 3\\
        $e_4$ & Rita &  {112} & { Mayabeque} & $a_4$ & Programaci\'on & 4\\
        $e_5$ & Carlos &  {113} & { Pinar del R\'io} & $a_3$ & \'Algebra & 3
    \end{tabular}

    \note{@NOTE alguien no depende completamente de la llave? Dependencias transitivas?}
\end{frame}

\begin{frame}{Recapitulando...}
    \begin{block}{Tercera Forma Normal}
        Un esquema relacional $R(U,F)$ est\'a en tercera forma normal
        (3FN), si est\'a en 2FN y los atributos no llaves son mutuamente independientes.
        
    \end{block}
\end{frame}

\begin{frame}{Descomposici\'on en 3ra Forma Normal}
    \vspace{-3mm}
    \begin{columns}[T]
        \begin{column}{0.68\linewidth}
            \begin{columns}[T]
                \begin{column}{0.6\textwidth}
                    \begin{center}
                        \textbf{Estudiante}\\[2mm]
        
                        \begin{tabular}{ccc}
                            \underline{\#E} & ENombre & Provincia\\[1mm]
                            \hline
                            $e_1$ & Juan & La Habana\\
                            $e_2$ & Mar\'ia & Matanzas\\
                            $e_3$ & Pedro & La Habana\\
                            $e_4$ & Rita & Mayabeque\\
                            $e_5$ & Carlos & Pinar del R\'io
                        \end{tabular}
                    \end{center}
                \end{column}

                \begin{column}{0.4\textwidth}
                    \begin{center}
                        \textbf{Provincia-Grupo}\\[2mm]
        
                        \begin{tabular}{cc}
                            \underline{Provincia} & Grupo\\[1mm]
                            \hline
                            La Habana & 111\\
                            Matanzas & 112 \\
                            Mayabeque & 112 \\
                            Pinar del R\'io & 113
                            
                        \end{tabular}
                    \end{center}
                \end{column}
                
            \end{columns}

            \begin{center}
                \textbf{Asignatura}\\[2mm]

                \begin{tabular}{cc}
                    \underline{\#A} & ANombre\\[1mm]
                    \hline
                    $a_1$ & An\'alisis\\
                    $a_2$ & L\'ogica \\
                    $a_3$ & \'Algebra\\
                    $a_4$ & Programaci\'on
                    
                \end{tabular}
            \end{center}
            
        \end{column}

        \begin{column}{0.3\linewidth}
            \vspace{6mm}
            \begin{center}
                \textbf{Evaluar}\\[2mm]

                \begin{tabular}{ccc}
                    \underline{\#E} & \underline{\#A} & Nota\\[1mm]
                    \hline
                    $e_1$ & $a_1$ & 3\\
                    $e_1$ & $a_2$ & 2\\
                    $e_1$ & $a_3$ & 4\\
                    $e_1$ & $a_4$ & 5\\
                    $e_3$ & $a_3$ & 4\\
                    $e_2$ & $a_1$ & 3\\
                    $e_2$ & $a_2$ & 3\\
                    $e_4$ & $a_2$ & 3\\
                    $e_4$ & $a_4$ & 4\\
                    $e_5$ & $a_3$ & 3\\
                \end{tabular}
            \end{center}
        \end{column}
    \end{columns}
    
    \note{@NOTE c\'omo obtenerla?}
\end{frame}


\begin{frame}{Eliminando dependencias problem\'aticas}
    \begin{block}{Cubrimiento minimal}
        Dado dos conjuntos de dependencias funcionales $F$ y $G$, se dice
        que $G$ es un cubrimiento minimal o cobertura irreducible
        de $F$ si se cumple que:
        \begin{enumerate}
            \item $G \equiv F$
            \item $G$ no contiene atributos redundantes
            \item $G$ no contiene dependencias redundantes
        \end{enumerate}
        
    \end{block}
\end{frame}


{
\setbeamertemplate{background} 
{
    \includegraphics[width=\paperwidth,height=\paperheight]{img/automate.jpg}
}
\begin{frame}
\end{frame}
}


\begin{frame}{Automatizando}
    \begin{block}{Algoritmo para obtener un cubrimiento minimal}
        
        \textbf{Entrada}: Un conjunto de DFs $F$ sobre un universo de atributos $U$.\\
        \textbf{Salida}: Un conjunto de DFs $G$, $G \equiv F$, sin atributos ni dependencias redundantes.
        \pause
        \textbf{M\'etodo}:
        \begin{enumerate}[<+->]
            \item A partir de $F$ construir un conjunto de DFs, $F'$, tal que el miembro derecho de cada DF sea un atributo simple.
            \item A partir de $F'$ construir un conjunto de DFs, $F''$, donde ning\'un determinante contiene atributos redundates; o sea,
            que para ninguna $X \to A$ en $F'$ y $Z \subset X$ se cumpla que
            $F' - \{X \to A\} \cup \{Z \to A\}$ sea equivalente a $F'$.
            \item A partir de $F''$ construir un conjunto de DFs, $F'''$, que no contenga dependencias
            redundantes; o sea, que para ninguna $X \to A$ en $F''$ el conjunto de
            dependencias funcionales $F'' - \{X \to A\}$ sea equivalente a $F''$.
        \end{enumerate}
    \end{block}
    % @TODO probar en el ejemplo cambiar los pasos 2 y 3 a ver si obtienes un ejemplo distinto; TLDR orden importa?
\end{frame}

\begin{frame}{Ejecutando el algoritmo}
    \begin{columns}[T]
        
        \begin{column}{0.25\linewidth}
            
            $AB \to C$\\
            $C \to A$\\
            $BC \to D$\\
            $ACD \to B$\\
            {\color<2-3>{red}$D \to EG$}\\
            $BE \to C$\\
            {\color<2-3>{red}$CG \to BD$}\\
            {\color<2-3>{red}$CE \to AG$}
            
        \end{column}
        \begin{column}{0.25\linewidth}
            
            \onslide<3->{

                $AB \to C$\\
                $C \to A$\\
                $BC \to D$\\
                {\color<5-6>{red}
                $ACD \to B$\\}
                {\color<3>{red}
                $D \to E$\\
                $D \to G$\\
                }
                $BE \to C$\\
                {\color<3>{red}
                $CG \to B$\\
                $CG \to D$\\
                $CE \to A$\\
                $CE \to G$
                }
            }

            
        \end{column}
        \begin{column}{0.25\linewidth}
            \onslide<6->{

            $AB \to C$\\
            $C \to A$\\
            $BC \to D$\\
            {\color<6>{red}
            $CD \to B$\\}
            $D \to E$\\
            $D \to G$\\
            $BE \to C$\\
            {\color<7>{red}
            $CG \to B$\\}
            $CG \to D$\\
            {\color<7>{red}
            $CE \to A$\\}
            $CE \to G$
        }
        \end{column}
        \begin{column}{0.25\linewidth}
            \onslide<8->{

            $AB \to C$\\
            $C \to A$\\
            $BC \to D$\\
            $CD \to B$\\
            $D \to E$\\
            $D \to G$\\
            $BE \to C$\\
            $CG \to D$\\
            $CE \to G$
        }
        \end{column}

    \end{columns}
    \vspace{5mm}

    \centering
    \only<4-5>{
        $D \to G \land CG \to B \models CD \to B$ 
    }
    \only<7>{
        $CG \to D \land CD \to B \models CG \to B$\\
        $C \to A \models CE \to A$
    }

    
\end{frame}

\begin{frame}{Algoritmo para obtener una descomposici\'on en 3FN}
    \textbf{Entrada:} Un esquema relacional $R(U,F)$, $F$ es un conjunto irreducible de dependencias funcionales.\\
    \textbf{Salida:} Una descomposici\'on $\rho = (R_1,R_2,...,R_n)$, tal que
    los esquemas relacionales $R_i(U_i,F_i)$ est\'an en 3FN con respecto
    a $\Pi_{R_i}(F)$, $\forall i = 1,...,n$.\\

    \pause
    \textbf{M\'etodo:}\begin{enumerate}
        \item<2-> \only<2>{Por cada dependencia funcional $X \to A_i$ en $F$ crear el esquema
        relacional $R_i(U_i,F_i)$ tal que $U_i = X \cup \{A_i\}$ y $F_i = \Pi_{R_i}(F)$. Si en $F$ se tiene \\$X \to A_1$, $X \to A_2$,..., $X \to A_k$ se puede
        utilizar un esquema relacional de la forma $R_j(U_j,F_j)$ con
        $U_j = X \cup \{A_1,A_2,...,A_k\}$ y $F_j = \Pi_{R_j}(F)$.}\only<3->{Crear un esquema por cada dependencia funcional. Preferiblemente, hacer que en el mismo esquema coincidan todas las dependencias funcionales con el mismo miembro izquierdo. Sean estos esquemas $R_1, R_2, ..., R_n$.}
        \item<4-> Si en $U$ existe alg\'un atributo que no est\'a contenido en ninguna dependencia
        funcional de $F$, este atributo puede formar un esquema relacional por s\'i mismo.
        \item<5-> Luego, $\rho = (R_1,R_2,...,R_n)$
    \end{enumerate}
\end{frame}

\begin{frame}{Obteniendo el dise\~no}
    \begin{columns}[T]
        \begin{column}{0.3\linewidth}
            \#E $\to$ ENombre\\
            {\color<2>{red}
            \#E $\to$ Grupo\\}
            \#E $\to$ Provincia\\
            \#A $\to$ ANombre\\
            \#E, \#A $\to$ \#E, \#A\\
            \#E, \#A $\to$ Nota\\
            Provincia $\to$ Grupo
        \end{column}

        \begin{column}{0.3\linewidth}
            \onslide<3>{
            \#E $\to$ ENombre\\
            \#E $\to$ Provincia\\
            \#A $\to$ ANombre\\
            \#E, \#A $\to$ \#E, \#A\\
            \#E, \#A $\to$ Nota\\
            Provincia $\to$ Grupo
            }
        \end{column}
        
    \end{columns}
    \vspace{5mm}
    \only<2>{
        \centering
        \#E $\to$ Provincia $\land$ Provincia $\to$ Grupo $\models$ \#E $\to$ Grupo
    }
\end{frame}


\begin{frame}{Obteniendo el dise\~no}
\centering
\begin{columns}[T]
    \begin{column}{0.48\linewidth}
        $R_1(U_1,F_1)$:\\
        \indent $U_1 = \{\textnormal{\#E, NombreE, Provincia}\}$\\
        \indent $F_1 = \pi_{U_1}(F)$\\[2mm]
        $R_2(U_2,F_2)$:\\
        \indent $U_2 = \{\textnormal{Provincia, Grupo}\}$\\
        \indent $F_2 = \pi_{U_2}(F)$
    \end{column}


    \begin{column}{0.48\linewidth}
        $R_3(U_3,F_3)$:\\
        \indent $U_3 = \{\textnormal{\#A, NombreA}\}$\\
        \indent $F_3 = \pi_{U_3}(F)$\\[2mm]
        $R_4(U_4,F_4)$:\\
        \indent $U_4 = \{\textnormal{\#E, \#A, Nota}\}$\\
        \indent $F_4 = \pi_{U_4}(F)$
    \end{column}
    
\end{columns}



\end{frame}

\begin{frame}{Obteniendo el dise\~no}
    \begin{columns}[T]
        \begin{column}{0.68\linewidth}
            \begin{columns}[T]
                \begin{column}{0.6\textwidth}
                    \begin{center}
                        \textbf{Estudiante}\\[2mm]
        
                        \begin{tabular}{ccc}
                            \underline{\#E} & ENombre & Provincia\\[1mm]
                            \hline
                            $e_1$ & Juan & La Habana\\
                            $e_2$ & Mar\'ia & Matanzas\\
                            $e_3$ & Pedro & La Habana\\
                            $e_4$ & Rita & Mayabeque\\
                            $e_5$ & Carlos & Pinar del R\'io
                        \end{tabular}
                    \end{center}
                \end{column}

                \begin{column}{0.4\textwidth}
                    \begin{center}
                        \textbf{Provincia-Grupo}\\[2mm]
        
                        \begin{tabular}{cc}
                            \underline{Provincia} & Grupo\\[1mm]
                            \hline
                            La Habana & 111\\
                            Matanzas & 112 \\
                            Mayabeque & 112 \\
                            Pinar del R\'io & 113
                            
                        \end{tabular}
                    \end{center}
                \end{column}
                
            \end{columns}

            \begin{center}
                \textbf{Asignatura}\\[2mm]

                \begin{tabular}{cc}
                    \underline{\#A} & ANombre\\[1mm]
                    \hline
                    $a_1$ & An\'alisis\\
                    $a_2$ & L\'ogica \\
                    $a_3$ & \'Algebra\\
                    $a_4$ & Programaci\'on
                    
                \end{tabular}
            \end{center}
            
        \end{column}

        \begin{column}{0.3\linewidth}
            \vspace{6mm}
            \begin{center}
                \textbf{Evaluar}\\[2mm]

                \begin{tabular}{ccc}
                    \underline{\#E} & \underline{\#A} & Nota\\[1mm]
                    \hline
                    $e_1$ & $a_1$ & 3\\
                    $e_1$ & $a_2$ & 2\\
                    $e_1$ & $a_3$ & 4\\
                    $e_1$ & $a_4$ & 5\\
                    $e_3$ & $a_3$ & 4\\
                    $e_2$ & $a_1$ & 3\\
                    $e_2$ & $a_2$ & 3\\
                    $e_4$ & $a_2$ & 3\\
                    $e_4$ & $a_4$ & 4\\
                    $e_5$ & $a_3$ & 3\\
                \end{tabular}
            \end{center}
        \end{column}
    \end{columns}
\end{frame}


% \begin{frame}{Aplicando el algoritmo}
%     \begin{columns}[T]
%         \begin{column}{0.3\linewidth}
%             {\color<2>{red}\#E, \#A} $\to$ ENombre\\
%             {\color<2>{red}\#E, \#A} $\to$ Grupo\\
%             {\color<2>{red}\#E, \#A} $\to$ Provincia\\
%             {\color<2>{red}\#E, \#A} $\to$ ANombre\\
%             {\color<2>{red}\#E, \#A} $\to$ Nota\\
%             {\color<4>{red}\#E} $\to$ ENombre\\
%             {\color<4>{red}\#E} $\to$ Grupo\\
%             {\color<4>{red}\#E} $\to$ Provincia\\
%             {\color<6>{red}\#A} $\to$ ANombre\\
%             {\color<8>{red}Provincia} $\to$ Grupo
%         \end{column}

%         \begin{column}{0.7\linewidth}
%             \only<3>{
%                 \begin{enumerate}
%                     \item $U = \{ \textnormal{\#E, ENombre,  Grupo, Provincia, \#A, ANombre, Nota}\}$
%                     \item $F = \{$ \\
%                     \hspace{10mm} $\textnormal{\#E} \to \textnormal{ENombre, Grupo, Provincia}$\\
%                     \hspace{10mm} $\textnormal{\#A} \to \textnormal{ANombre}$\\
%                     \hspace{10mm} $\textnormal{\#E,\#A} \to \textnormal{\#E,\#A}$\\
%                     \hspace{10mm} $\textnormal{\#E, \#A} \to \textnormal{Nota}$\\
%                     \hspace{10mm} $\textnormal{Provincia} \to \textnormal{Grupo}$\\
%                     $\}$
%                     \item Definimos el esquema relacional $\textbf{Evaluaciones}(U,F)$ con llave \#E, \#A 
%                 \end{enumerate}

%             }
%             \only<5>{
%                 $U_2 = \{\textnormal{\#E, ENombre, Grupo, Provincia}\}$\\
%                 $F_2 = \{\textnormal{\#E} \to \textnormal{ENombre, Grupo, Provincia,ANombre}$\\
%                 $\textnormal{Provincia} \to \textnormal{Grupo}$\} 
%             }
%         \end{column}
%     \end{columns}
% \end{frame}







% \begin{frame}{Formas normales}
%     \begin{block}{Forma Normal de Boyce-Codd}
%         Un esquema relacional $R(U,F)$ est\'a en BCFN
%         si cada uno de sus determinantes constituye una llave
%         o (superllave) candidata.
        
%     \end{block}
% \end{frame}

% \begin{frame}{3FN vs BCFN}

%     \begin{itemize}
%         \item 3FN Puede existir redundancia entre atributos que pertenecen a alguna llave
%         \item 3FN $\neq$ BCFN se ponde de manifiesto cuando en la
%         relaci\'on existe m\'as de una llave candidata, que sean compuestas y se solapen.
%     \end{itemize}
% \end{frame}