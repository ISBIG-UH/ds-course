\begin{frame}{Objetivos de la asignatura}
    \begin{block}{Proporcionar un conjunto de m\'etodos y herramientas para:}
        \only<2>{
        \begin{tikzpicture}[overlay, remember picture]
            \draw[red, line width=2pt] (current page.south west) -- (current page.north east);
            \draw[red, line width=2pt] (current page.north west) -- (current page.south east);
        \end{tikzpicture}

        }
        \begin{itemize}
            \item Dise\~nar e implementar bases de datos correctas
            \item Evaluar la calidad de bases de datos espec\'ificas
            \item Identificar la vigencia del modelo relacional y sus limitaciones
            \item Reconocer casos de uso para bases de datos no relacionales
        \end{itemize}
    \end{block}
\end{frame}

\begin{frame}{Objetivos de la asignatura}
    \begin{columns}
        \begin{column}{.6\textwidth}
            \begin{block}{Proporcionar un conjunto de m\'etodos y herramientas para:}
                \begin{itemize}
                    \item<2-> Extraer una modelaci\'on conceptual de una base de datos
                    \item<4-> Obtener informaci\'on relevante de una base de datos
                    % @audit ver si anyadimos ma's
                \end{itemize}
            \end{block}
        \end{column}
        \begin{column}{.4\textwidth}
            \includegraphics<3,5>[width=\textwidth]{img/all-types-of-dbs.png}
        \end{column}
    \end{columns}

    \note<3,5>{@NOTE relacionales y no relacionales}
\end{frame}