  



\begin{frame}{Modelando conjuntos}
    \begin{block}{}
        \textcolor{orange}{Conjunto de entidades}: Conjunto de objetos que se puedan identificar
        en el escenario que se desea representar y que tienen cierto
        significado para el usuario.

        \vspace{3mm}
        Conjunto: JUGADOR = \{Juan, Marcos, Mar\'ia,...\}
        \vspace{3mm}

        Representaci\'on gr\'afica:

        \vspace{3mm}
        \centering
        \begin{tikzpicture}
            \node[entity,minimum width=2.3cm, minimum height=1.2cm] (jugador) at (3.2,0) {JUGADOR};
        \end{tikzpicture}
    \end{block}

    \note{@NOTE esta representaci\'on gr\'afica se trata as\'i por convenio}
\end{frame}

\begin{frame}{Modelando interrelaciones}
    \begin{block}{}
        \textcolor{blue}{Conjuntos de interrelaciones}: Un conjunto
        de concatenaciones de instancias tomadas de los conjuntos
        de entidades que se relacionan.

        \vspace{3mm}

        Conjunto: PERTENECER = \{(Juan, TheWarriors), (Pedro, bravehearts), ...\}

        \vspace{3mm}

        Representaci\'on gr\'afica:

        \vspace{3mm}

        \centering
        \begin{tikzpicture}
            \node[entity,minimum width=2.3cm, minimum height=1.2cm] (jugador) at (3.4,0) {JUGADOR};
            \node[entity,minimum width=2.3cm, minimum height=1.2cm] (clan) at (13,0) {CLAN};
            \node[relationship, aspect=2] at (8.2,0) {PERTENECER} edge(jugador) edge(clan);
        \end{tikzpicture}
        
        \vspace {3mm}

        \onslide<2>{
        \Large \textcolor{red}{¿Qu\'e instancias pertenecen a este conjunto?}
        }

    \end{block}

    \note<1>{@NOTE rombo para $n \leq 4$. Pent\'agono pa $n = 5$, etc}
\end{frame}




