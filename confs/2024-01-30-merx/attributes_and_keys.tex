\begin{frame}{A\~nadiendo estructura}
    De cada {\color<2>{orange}jugador} se conoce su {\color<2>{attr}carnet de identidad}, 
    {\color<2>{attr}nombre}, 
    {\color<2>{attr}nivel}, la 
    {\color<2>{attr}cantidad de trofeos que tiene actualmente} y el 
    {\color<2>{attr}m\'aximo de trofeos que ha alcanzado}.
\end{frame}


% \begin{frame}{Atributo}

%     \begin{columns}[T]

%         \begin{column}{0.5\linewidth}
%             \begin{tikzpicture}
%                 \tikzstyle{every entity} = [minimum width=2cm, minimum height=0.8cm]
%                 \node[entity] (jugador) at (0,-4) {\small JUGADOR};
%                 \node[entity] (email) at (-2,2) {\small CI};
%                 \node[entity] (apodo)at (2,2) {\small NOMBRE};
        
%                 \node[relationship, aspect=2] (identificar) at (-2,-1) {\small IDENTIFICAR} edge(jugador) edge(email);
%                 \node[relationship] (tener) at (2,-1) {\small TENER} edge(jugador) edge(apodo);
        
%                 \node at (-2.4,1.3) {$1,{\color<2>{red}1}$};
%                 \onslide<2>{\draw[->,thick,color=red] (-2.6,0.6) -- (-2.3,1.1);}
%                 \node at (2.4,1.3) {${\color<5>{red}0},{\color<2>{red}1}$};
%                 \onslide<2>{\draw[->,thick,color=red] (3,0.6) -- (2.7,1.1);}
%                 \onslide<5>{\draw[->,thick,color=red] (2.6,0.5) -- (2.3,1.1);}

        
%                 \node at (-0.9,-3.4) {$1,{\color<3>{red}1}$};
%                 \onslide<3>{\draw[->,thick,color=red] (-1.5,-2.8) -- (-0.8,-3.2);}
%                 \node at (0.9,-3.4) {$1,{\color<4>{red}\ast}$};
%                 \onslide<4>{\draw[->,thick,color=red] (1.7,-2.6) -- (1.2,-3.2);}
        
        
%             \end{tikzpicture}
%         \end{column}

%         \begin{column}{0.48\linewidth}
%             \vspace{20mm}
%             \only<1>{
%                 \begin{block}{Definici\'on formal}
%                     Es una interrelaci\'on entre el conjunto
%                     de entidades de inter\'es y un conjunto de entidades que representa el atributo.
%                 \end{block}
%             }
%             \only<2-5>{
%                 \begin{alertblock}{Consideraciones importantes}
%                     \only<2> {Las cardinalidades m\'aximas en el extremo de los atributos siempre es 1}
%                     \only<3> {Existen atributos cuyo valor es \'unico para cada instancia del conjunto de entidades de inter\'es}
%                     \only<4> {Existen atributos que pueden tener el mismo valor para varias instancias del conjunto de entidades de inter\'es}
%                     \only<5> {Una instancia puede no tener valor asociado para un atributo, en cuyo caso el atributo es nulo}     
%                 \end{alertblock}

%             }
           
%         \end{column}
        
%     \end{columns}
    
% \end{frame}

\begin{frame}{Atributo}
    \begin{columns}[T]
        \begin{column}{0.5\linewidth}
            \centering
            \begin{tikzpicture}[node distance=6em]
                \tikzstyle{every entity} = [minimum width=2cm, minimum height=0.8cm]

                \node[entity] (jugador) {JUGADOR}
                    [sibling distance=3cm]
                    child {node[attribute] [above left of=jugador] {\tiny TROFEOS MAX}}
                    child {node[attribute] [above right of=jugador]{\tiny TROFEOS}}
                    child {node[attribute] [above of=jugador] {\tiny NIVEL}}
                    child {node[attribute] [right of=jugador]{\tiny NOMBRE}}
                    child {node[attribute] (ci) [left of=jugador] {\tiny \underline{CI}}};

                \only<3>{

                    \draw[<-,thick,color=red] (ci.south east) -- (-1.4,-1);
                }
            \end{tikzpicture}

          
        \end{column}
        \begin{column}{0.48\linewidth}
            \vspace{10mm}
            \begin{block}{Definici\'on informal}
                Es una propiedad de un tipo de entidades.
            \end{block}
        \end{column}
    \end{columns}

    \only<2>{
        \vspace{5mm}
        \small {JUGADOR = \{
        (98012300205, Juan, 1, 1800, 2300), 
        (97041223987, Pedro, 3, 1600, 1800),
        (99072392022, Mar\'ia, 5, 1900, 2500),... 
        \}}
    }
   
    \only<3>{
    \vspace{5mm}

    \centering
    ¿Por qu\'e el carnet de identidad est\'a subrayado?
    }
\end{frame}


\begin{frame}{Atributo llave}
    \begin{block}{Problema}
        ¿C\'omo saber si en los conjuntos de entidades o conjuntos de interrelaciones existen
        instancias repetidas?
    \end{block}

    \begin{block}<2->{Soluci\'on}
        \begin{itemize}
            \item<2-> Una llave es un valor que siempre puede utilizarse de forma
            un\'ivoca para identificar una instancia dentro de un conjunto de instancias.
            \item<3-> La llave de un conjunto de entidades es una concatenaci\'on de una selecci\'on de sus atributos.
            \item<4-> La llave
            de un conjunto de interrelaciones es una concatenaci\'on de las llaves de
            los conjuntos entidades que intervienen en la relaci\'on.
        \end{itemize}
       
    \end{block}
\end{frame}

\begin{frame}{Estructurando los conjuntos de entidades}
    \begin{block}{}
        Las {\color<2->{orange}cartas} tienen un {\color<2->{attr}identificador}, un {\color<2->{attr}nombre}, una {\color<2->{attr}descripci\'on}, 
        un {\color<2->{attr}costo de elixir}  y una {\color<2->{attr}calidad} (común,
        especial, épica o legendaria).
    \end{block}
    \vspace{5mm}

    \onslide<3>{
    \centering
    \begin{tikzpicture}[node distance=6em]
        \tikzstyle{every entity} = [minimum width=2cm, minimum height=0.8cm]

        \node[entity] (carta) {CARTA}
            [sibling distance=3cm]
            child {node[attribute] [above left of=carta] {\tiny CALIDAD}}
            child {node[attribute] [above right of=carta]{\tiny DESC.}}
            child {node[attribute] [above of=carta] {\tiny NOMBRE}}
            child {node[attribute] [right of=carta]{\tiny COSTO}}
            child {node[attribute] [left of=carta] {\tiny \underline{CARTA\_ID}}};
    \end{tikzpicture}
    }
\end{frame}

\begin{frame}{Estructurando los conjuntos de entidades}
    \begin{block}{}
        De los {\color<2->{orange}clanes} se conoce su {\color<2->{attr}identificador}, {\color<2->{attr}nombre}, una {\color<2->{attr}regi\'on}, 
        un {\color<2->{attr}tipo} (solo invitaci\'on o abierto)  y una {\color<2->{attr}cantidad m\'inima de trofeos} para entrar.
    \end{block}
    \vspace{5mm}

    \onslide<3>{
    \centering
    \begin{tikzpicture}[node distance=6em]
        \tikzstyle{every entity} = [minimum width=2cm, minimum height=0.8cm]

        \node[entity] (clan) {CLAN}
            [sibling distance=3cm]
            child {node[attribute] [above left of=clan] {\tiny TROFEOS MIN.}}
            child {node[attribute] [above right of=clan]{\tiny TIPO}}
            child {node[attribute] [above of=clan] {\tiny NOMBRE}}
            child {node[attribute] [right of=clan]{\tiny REGI\'ON}}
            child {node[attribute] [left of=clan] {\tiny \underline{CLAN\_ID}}};
    \end{tikzpicture}
    }

    \note<3>{@NOTE dudas?}
\end{frame}

% @TODO corta aki' la conf y ponle ejercicios. 

