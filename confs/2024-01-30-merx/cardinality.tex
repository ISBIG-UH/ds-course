
\begin{frame}{Restricciones sobre interrelaciones}

    \begin{block}{Cardinalidad de una interrelaci\'on}
        Cantidad de instancias en un conjunto de entidades que
        puede estar relacionado con una \'unica instancia en el otro
        conjunto de entidades.
    \end{block}

    \pause

    \begin{block}{¿C\'omo determinar la cardinalidad de una interrelaci\'on?}
        Por cada conjunto de entidades en un extremo de la interrelaci\'on: \begin{enumerate}
            \item Fijar una \'unica entidad en el conjunto de entidades restante
            \item Calcular la cantidad de instancias relacionadas con las entidades fijadas
        \end{enumerate}
    \end{block}
\end{frame}

% One to many relationship example
\begin{frame}
    {
    \only<-9>{Determinar la cardinalidad de una interrelaci\'on. Ejemplo}
    \only<10>{Interrelaciones de uno a muchos}
    }

    \begin{overlayarea}{\linewidth}{\textheight}
        \vspace{10mm}
        \begin{center}
            \Large{
                \textcolor{orange}{jugador} \hspace{10mm} \textcolor{blue}{pertenecer} \hspace{10mm} \textcolor{orange}{clan}
            } 
        \end{center}
        \vspace{3mm}

        \centering
        \begin{tikzpicture}
            \tikzstyle{every entity} = [minimum width=2.3cm, minimum height=1.2cm]
            \node[entity] (jugador) at (0,0) {JUGADOR};
            \node[entity] (clan) at (9,0) {CLAN};
            \node[relationship, aspect=2] (pertenecer) at (4.5,0) {PERTENECER} edge(jugador) edge(clan);
            \onslide<4-8>{\node at (7.5,0.2) {{\color<4,8>{red}{1}}};}
            \only<4>{\draw[->,thick,color=red](7,0.7) -- (7.4,0.4);}
            \onslide<7-8>{\node at (1.5,0.2) {{\color<7,8>{red}{$\ast$}}};}
            \only<7>{\draw[->,thick,color=red](1.9,0.8) -- (1.6,0.4);}
            \onslide<9-10>{\node at (7.5,0.2) {${\color<9>{red}{0}}, 1$};}
            \onslide<9-10>{\node at (1.5,0.2) {${\color<9>{red}{1}}, \ast$};}

        \end{tikzpicture}

        \only<2-4>{
            \begin{block}{Cardinalidad en el extremo CLAN}
                \begin{enumerate}
                    \item Fijamos una \'unica entidad en el conjunto JUGADOR
                    \item<3-4> ¿Un jugador a cu\'antos clanes puede pertenecer?
                    \item<4> Un jugador puede pertenecer a un \textcolor{red}{\'unico} clan
                \end{enumerate}
            \end{block}
        }

        \only<5-7>{
            \begin{block}{Cardinalidad en el extremo JUGADOR}
                \begin{enumerate}
                    \item Fijamos una \'unica entidad en el conjunto CLAN
                    \item<6-7> ¿A un clan cu\'antos jugadores pueden llegar a pertenecer?
                    \item<7> A un clan pueden pertenecer \textcolor{red}{muchos} jugadores
                \end{enumerate}
            \end{block}
        }

        \only<8>{
            \vspace{5mm}

            \centering
            \Large{\textcolor{red}{ Cardinalidad m\'axima de una interrelaci\'on (posibilidad)}}
        
        }

        \only<9>{
            \vspace{5mm}

            \centering
            {\Large \textcolor{red}{ Cardinalidad m\'inima de una interrelaci\'on (opcionalidad)}}
            \vspace{2mm}

            ¿Un jugador al menos a cu\'antos clanes debe pertenecer?

            ¿A un clan al menos cu\'antos jugadores deben pertenecer?
        }

        \only<10>{
            \vspace{5mm}
            Si la cardinalidad m\'axima en una direcci\'on es 1 y en la otra
            es mayor que 1 se dice que la interrelaci\'on 
            es de \textcolor{red}{uno a muchos} (o viceversa, de muchos a uno) 
            y es denotada por \textcolor{red}{ $1 : \ast$} (o viceversa, \textcolor{red}{$\ast : 1$}). 
        }

    \end{overlayarea}
\end{frame}


% One to one relationship example
\begin{frame}{
    \only<-7>{Determinar la cardinalidad de una interrelaci\'on. Otro ejemplo}
    \only<8>{Interrelaciones de uno a uno}
    }

    \begin{overlayarea}{\linewidth}{\textheight}
        \vspace{10mm}
        \begin{center}
            \Large{
                \textcolor{orange}{jugador} \hspace{10mm} \textcolor{blue}{ser l\'ider} \hspace{10mm} \textcolor{orange}{clan}
            }
        \end{center}
        \vspace{3mm}

        \centering
        \begin{tikzpicture}
            \tikzstyle{every entity} = [minimum width=2.3cm, minimum height=1.2cm]
            \node[entity] (jugador) at (0,0) {JUGADOR};
            \node[entity] (clan) at (9,0) {CLAN};
            \node[relationship, aspect=2] (serlider) at (4.5,0) {SER\_L\'IDER} edge(jugador) edge(clan);
            \onslide<4->{\node at (7.5,0.2) {$0, 1$};}
            \onslide<7->{\node at (1.5,0.2) {$1, 1 $};}
        \end{tikzpicture}


        \only<2-4>{
            \begin{block}{Cardinalidad en el extremo CLAN}
                \begin{enumerate}
                    \item Fijamos una \'unica entidad en el conjunto JUGADOR
                    \item<3-4> ¿Un jugador de cu\'antos clanes puede ser l\'ider?
                    \item<4> Un jugador puede ser l\'ider de ning\'un clan o de uno solo
                \end{enumerate}
            \end{block}
        }

        \only<5-7>{
            \begin{block}{Cardinalidad en el extremo JUGADOR}
                \begin{enumerate}
                    \item Fijamos una \'unica entidad en el conjunto CLAN
                    \item<6-7> ¿De un clan cu\'antos jugadores pueden ser l\'ider?
                    \item<7> Un clan tiene un y solo un l\'ider
                \end{enumerate}
            \end{block}
        }

        \only<8>{
            \vspace{5mm}

            Si la cardinalidad m\'axima en ambas direcciones de la interrelaci\'on es 1
            se dice que la interrelaci\'on es de \textcolor{red}{uno a uno} y es denotada por \textcolor{red}{$1:1$}.
        }
    \end{overlayarea}
\end{frame}


% Many to many relationship example
\begin{frame}{
    \only<-7>{Determinar la cardinalidad de una interrelaci\'on. S\'i, otro m\'as}
    \only<8>{Interrelaciones de muchos a muchos}
    }
    \begin{overlayarea}{\textwidth}{\textheight}
        \vspace{10mm}
        \begin{center}
            \Large{
                \textcolor{orange}{jugador} \hspace{10mm} \textcolor{blue}{coleccionar} \hspace{10mm} \textcolor{orange}{carta}
            }
        \end{center}
        \vspace{3mm}
    
        \centering
        \begin{tikzpicture}
            \tikzstyle{every entity} = [minimum width=2.3cm, minimum height=1.2cm]
            \node[entity] (jugador) at (0,0) {JUGADOR};
            \node[entity] (carta) at (9,0) {CARTA};
            \node[relationship, aspect=2] (coleccionar) at (4.5,0) {COLECCIONAR} edge(jugador) edge(carta);
            \onslide<4->{\node at (7.5,0.2) {$1,\ast$};}
            \onslide<7->{\node at (1.5,0.2) {$0,\ast$};}
        \end{tikzpicture}
    
        \only<2-4>{
            \begin{block}{Cardinalidad en el extremo CARTA}
                \begin{enumerate}
                    \item Fijamos una \'unica entidad en el conjunto JUGADOR
                    \item<3-4> ¿Un jugador cu\'antas cartas puede coleccionar?
                    \item<4> Un jugador puede llegar a coleccionar todas las cartas y debe de tener al menos una
                \end{enumerate}
            \end{block}
        }
    
        \only<5-7>{
            \begin{block}{Cardinalidad en el extremo JUGADOR}
                \begin{enumerate}
                    \item Fijamos una \'unica entidad en el conjunto CARTA
                    \item<6-7> ¿Una carta cu\'antos jugadores pueden coleccionarla?
                    \item<7> Muchos jugadores pueden coleccionar una misma carta y pueden existir cartas muy raras que ninguno tenga
                \end{enumerate}
            \end{block}
        }
    
        \only<8>{
        \vspace{5mm}
            
            Si las \textcolor{red}{cardinalidades m\'aximas} en ambas direcciones
            son mayores que 1 se dice que la interrelaci\'on 
            es \textcolor{red}{muchos a muchos} y es \textcolor{red}{denotada por $\ast : \ast$}. 
        }
    \end{overlayarea}

  
\end{frame}


