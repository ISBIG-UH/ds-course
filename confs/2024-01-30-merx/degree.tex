
\begin{frame}{Tipos de interrelaciones}
    \begin{block}{Grado de una interrelaci\'on}
        Cantidad de conjuntos de entidades entre los que se establece la interrelaci\'on
    \end{block}

    \begin{block}{Tipos existentes}
        \begin{itemize}
            \item Unaria o recursiva
            \item Binaria
            \item Ternaria
            \item $n$-aria con $n > 3$
        \end{itemize}
    \end{block}
\end{frame}



\begin{frame}{Interrelaci\'on unaria}
    \begin{center}
        \Large{\textcolor{orange}{jugador} \hspace{10mm} \textcolor{blue}{ser amigo} \hspace{10mm} \textcolor{orange}{jugador}}
    \end{center}
    \vspace{5mm}

    \centering
    \begin{tikzpicture}
        \node[entity, minimum width=2.3cm, minimum height=1.2cm] (jugador) at (0,0) {JUGADOR};
        \node[relationship, aspect=2] (seramigo) at (0,-2.5) {SER\_AMIGO};
        \draw (seramigo.east) -- (2,-2.5) -- (2,0) -- (jugador.east);
        \draw (seramigo.west) -- (-2,-2.5) -- (-2,0) -- (jugador.west);
        \node at (-1.5,0.2) {$0 , \ast$};
        \node at (1.5,0.2) {$0,\ast$};

    \end{tikzpicture}
\end{frame}


\begin{frame}{Interrelaci\'on binaria}
    \begin{center}
        \Large{
            \textcolor{orange}{jugador} \hspace{10mm} \textcolor{blue}{coleccionar} \hspace{10mm} \textcolor{orange}{carta}
        }
    \end{center}
    \vspace{3mm}

    \centering
    \begin{tikzpicture}
        \tikzstyle{every entity} = [minimum width=2.3cm, minimum height=1.2cm]
        \node[entity] (jugador) at (0,0) {JUGADOR};
        \node[entity] (carta) at (9,0) {CARTA};
        \node[relationship, aspect=2] (coleccionar) at (4.5,0) {COLECCIONAR} edge(jugador) edge(carta);
        \node at (7.5,0.2) {$1,\ast$};
        \node at (1.5,0.2) {$0,\ast$};
    \end{tikzpicture}
\end{frame}

\begin{frame}{Interrelaci\'on ternaria}
    \begin{overlayarea}{\linewidth}{\textheight}
        \vspace{5mm}
        \begin{center}
            \Large{
                \textcolor{orange}{jugador} \hspace{10mm} \textcolor{blue}{donar} \hspace{10mm} \textcolor{orange}{carta} \hspace{10mm} \textcolor{orange}{clan}
            }
        \end{center}
        \vspace{3mm}
    
        \centering
        \begin{tikzpicture}
            \tikzstyle{every entity} = [minimum width=2.3cm, minimum height=1.2cm]
            \node[entity] (jugador) at (0,0) {JUGADOR};
            \node[entity] (carta) at (9,0) {CARTA};
            \node[entity] (clan) at (4.5,-3) {CLAN};
            \node[relationship, aspect=2] (donar) at (4.5,0) {DONAR} edge(jugador) edge(carta) edge(clan);
            \onslide<3->{\node at (7.5,0.2) {$0,\ast$};}
            \onslide<5->{\node at (1.5,0.2) {$0,\ast$};}
            \onslide<7->{\node at (4.9,-2.2) {$0,1$};}
        \end{tikzpicture}
    
        \only<2>{
            \begin{block}{Cardinalidad en el extremo CARTA}
                ¿Un jugador puede donar a un clan cu\'antas cartas?
            \end{block}
        }
    
        \only<4>{
            \begin{block}{Cardinalidad en el extremo JUGADOR}
                ¿Una carta es donada a un clan por cu\'antos jugadores?
            \end{block}
        }
        \only<6>{
            \begin{block}{Cardinalidad en el extremo CLAN}
                ¿Un jugador puede donar una carta a cu\'antos clanes?
            \end{block}
        }
    \end{overlayarea}
\end{frame}

\begin{frame}{Interrelaci\'on unaria \only<2->{¿Seguros?}}
    \begin{overlayarea}{\linewidth}{\textheight}
        \vspace{6mm}
        \begin{center}
            \Large{
                \textcolor{orange}{jugador} \hspace{10mm} \textcolor{blue}{enfrentar} \hspace{10mm} \textcolor{orange}{jugador}
            }
        \end{center}
    
        \vspace{5mm}
    
        \centering
        \begin{tikzpicture}
            \node[entity, minimum width=2.3cm, minimum height=1.2cm] (jugador) at (0,0) {JUGADOR};
            \node[relationship, aspect=2] (enfrentar) at (0,-2.5) {ENFRENTAR};
            \draw (enfrentar.east) -- (2,-2.5) -- (2,0) -- (jugador.east);
            \draw (enfrentar.west) -- (-2,-2.5) -- (-2,0) -- (jugador.west);
            \node at (-1.5,0.2) {$0 , \ast$};
            \node at (1.5,0.2) {$0,\ast$};
    
            
            \only<3>{
                \node[align=left] at (6,-1) {
                    \{{\color<3>{red}(Pedro, Juan)}, {\color<3>{red}(Pedro, Juan)}, ... \}
                };
            }
        \end{tikzpicture}
    
    
        \vspace{3mm}
    
        \only<2>{¿Qu\'e ocurre si dos jugadores se enfrentan m\'as de una vez?}
        \only<3>{\textcolor{red}{Se crear\'ian instancias duplicadas en el conjunto de interrelaciones. Imposible}}
    \end{overlayarea}
\end{frame}

\begin{frame}{Interrelaciones en el tiempo}
    
    \centering
    \begin{tikzpicture}
        \tikzstyle{every entity} = [minimum width=2.3cm, minimum height=1.2cm]
        \node[entity] (jugador) at (0,0) {JUGADOR};
        \node[entity] (fecha) at (9,0) {FECHA};
        \node[relationship, aspect=2] (enfrentar)at (4.5,0) {ENFRENTAR} edge(fecha);
        \draw (jugador.south) -- (0,-1.2) -- (4.5,-1.2) -- (enfrentar.south); 
        \draw (jugador.north) -- (0,1.2) -- (4.5,1.2) -- (enfrentar.north); 

        \node at (-0.4,0.9) {$0,1$};
        \node at (-0.4,-0.9) {$0,1$};
        \node at (7.5,0.2) {$0,\ast$};
    \end{tikzpicture}

    \vspace{5mm}

    \centering
        \{
            (Pedro, Juan, 24/02/23-14:55),
            (Pedro, Juan, 24/02/23-16:00),
            ...
        \}


\end{frame}

