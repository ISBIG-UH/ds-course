\begin{frame}
    \frametitle{Ejercicios}
    \framesubtitle<1>{1. Suministro de productos}
    \framesubtitle<2>{2. Pr\'estamo de libros}
    \framesubtitle<3>{3. Peque\~na cafetería}

    \only<1>{
        Se desea modelar el suministro de productos de una tienda. De cada suministrador se conoce su identificador, su nombre, su tipo y el municipio al que pertenece. De cada producto se conoce su c\'odigo, su nombre, su precio y la unidad de medida que le corresponde. Un suministrador suministra un producto en una cierta cantidad. Cada suministrador puede suministrar varios productos. Cada producto puede ser suministrado por varios suministradores.
    }
    \only<2>{
        En una librer\'ia se prestan libros a sus miembros. De cada libro se tiene registrado su t\'itulo, autor, ISBN, a\~no de publicaci\'on y g\'enero. Los miembros son usuarios de la librer\'ia y de ellos se conoce un identificador \'unico, el nombre, direcci\'on, tel\'efono y correo. Cuando un libro es prestado a un miembro se registra la fecha en la que fue emitido el pr\'estamo, la fecha en la que debe ser devuelto el libro y la fecha en la que realmente este se devolvi\'o. Cada pr\'estamo est\'a constituido por un solo libro y es emitido a nombre de un solo miembro.
    }
    \only<3>{
        En una pequeña cafetería de barrio los clientes visitan para comprar café y pasteles. Cada cliente posee una identificación única y puede acumular puntos de fidelidad. Se realizan pedidos, cada uno con un ID de pedido único, que comprende varios productos como café y pasteles. Los productos tienen sus propios ID, nombres, tipos y precios. Los empleados, incluidos barman y gerentes, son responsables de procesar los pedidos. Cada empleado tiene un ID, nombre, rol y turnos asignados. Los pedidos se rastrean por su estado, monto total y método de pago. 
    }

    \note<2>{@NOTE el ISBN es un c\'odigo \'unico para identificar un libro}
\end{frame}