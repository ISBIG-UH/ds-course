\begin{frame}{Respondiendo preguntas}

    \centering
    \Huge\textcolor{blue3}{Lenguajes de consulta}
    
\end{frame}


\begin{frame}{Lenguajes de consulta}
    \begin{block}{Definici\'on}
        Son aquellos lenguajes utilizados para definir solicitudes de recuperaci\'on
        sobre los datos almacenados en una base de datos
        
    \end{block}

    \begin{block}{Consulta}
        Una solicitud de recuperación, es decir, una expresión relacional o una declaración que solicita la
        evaluación de tal expresión
    \end{block}
\end{frame}

\begin{frame}{Caso de estudio}

    \centering
    \begin{tikzpicture}[node distance=6em]
        \tikzstyle{every entity} = [minimum width=2cm, minimum height=1.2cm]
        \node[entity] (jugador) {JUGADOR}
            [sibling distance=3cm]
            child {node[attribute] [above right of=jugador] {\tiny TROFEOS MAX}}
            child {node[attribute] [above of=jugador] {\tiny TROFEOS}}
            child {node[attribute] [above left of=jugador] {\tiny NOMBRE}}
            child {node[attribute] [left of=jugador] {\underline{\tiny \#J}}}
            child {node[attribute] [below left of=jugador] {\tiny NIVEL}}
            ;
      
        \node[entity] (clan) at (8,0) {CARTA}
        [sibling distance=3cm]
        child {node[attribute] [right of=clan] {\underline{\tiny \#C}}}
        child {node[attribute] [above of=clan] {\tiny COSTO}}
        child {node[attribute] [above left of=clan] {\tiny DESC.}}
        child {node[attribute] [above right of=clan] {\tiny NOMBRE}}
        child {node[attribute] [below right of=clan] {\tiny CALIDAD}};
        
      
        \node[relationship,aspect=2] (pertenecer) at (4,0) {COLECCIONAR};
        \draw (pertenecer.east) -- (clan.west) node[above left] {$1,\ast$};
        \draw (pertenecer.west) -- (jugador.east) node[above right] {$0,\ast$};
    \end{tikzpicture}
\end{frame}




\begin{frame}{\'Algebra relacional}

    \begin{block}{Consultas}
        \begin{enumerate}
            \item<2-> Obtener el nombre de todos los jugadores cuyo nivel es al menos 10.
            \onslide<3->{$$r := \pi_{\text{\#J, Nombre}} (\text{Jugador}\,\sigma\, (\text{Nivel} \geq 10))$$}
            \item<4-> Obtener el nombre de todas las cartas cuya calidad sea \'epica.
            \onslide<5->{$$r := \pi_{\text{\#C, Nombre}} (\text{Carta}\,\sigma\, (\text{Calidad} = \text{"\'Epica"}))$$}
        \end{enumerate}
    \end{block}

\end{frame}


% \begin{frame}{\'Algebra relacional}

%     \begin{block}{Consultas}
%         \begin{enumerate}
%             \item Obtener el nombre de todos los jugadores cuyo nivel es al menos 10.
%             $$\pi_{\text{\#J, Nombre}} (\text{Jugador}\,\sigma\, (\text{Nivel} >= 10))$$ 
%             \item Obtener el nombre de todas las cartas cuya calidad sea \'epica.
%             $$\pi_{\text{\#C, Nombre}} (\text{Carta}\,\sigma\, (\text{Calidad} = \text{\'Epica}))$$
%         \end{enumerate}
%     \end{block}

% \end{frame}


\begin{frame}{\'Algebra relacional}

    \begin{block}{Consultas}
        \begin{enumerate}
            \setcounter{enumi}{2}
            \item Obtener el nombre de todos los jugadores que tienen la carta con identificador 2.
            $$ r_1 = \pi_{\text{\#J}, \text{Nombre}}((\text{Jugador} \Join \text{Coleccionar})\,\sigma\,(\text{\#C}=2)) $$
            $$ r_2 = \pi_{\text{\#J}, \text{Nombre}}(\text{Jugador} \Join (\text{Coleccionar}\,\sigma\,(\text{\#C}=2))) $$
        \end{enumerate}
        \vspace{5mm}

        \centering
        \large ¿Cu\'al es mejor?
        
        
    \end{block}
\end{frame}


\begin{frame}{El problema del \'algebra relacional}

   
    $$ \pi_{\text{\#J}, \text{Nombre}}((\text{Jugador} \Join \text{Coleccionar})\,\sigma\,(\text{\#C}=2)) $$

    \begin{block}{¿En qu\'e orden se ejecutan las operaciones?}
        \onslide<2->{
        \begin{enumerate}
            \item $\text{Jugador} \Join \text{Coleccionar}$\hspace{4mm} \onslide<3->{$O(|\text{Jugador}| \times |\text{Coleccionar}|)$}
            \item $(\text{Jugador} \Join \text{Coleccionar})\,\sigma\,(\text{\#C} = 2)$\hspace{4mm} \onslide<4->{$O(|\text{Coleccionar}|)$}
            \item $\pi_{\text{\#J}, \text{Nombre}}((\text{Jugador} \Join \text{Coleccionar})\,\sigma\,(\text{\#C}=2))$ \hspace{4mm} \onslide<5->{$O(|\text{Jugador}|)$}
        \end{enumerate}
        }

        \begin{center}
            \onslide<6->{
            \vspace{5mm}
            $
            O(|\text{Jugador}| \times |\text{Coleccionar}| + |\text{Coleccionar}| + |\text{Jugador}|)
            $
            }
            
            \onslide<7>{
            \vspace{2mm}
            $
            = O(|\text{Jugador}|^2 \times |\text{Carta}| + |\text{Coleccionar}| + |\text{Jugador}|)
            $\\[2mm]
            dado que $|\text{Coleccionar}| = O(|\text{Jugador}| \times |\text{Carta}|)$
            }

    
        \end{center}
    \end{block}
         
    
\end{frame}


\begin{frame}{El problema del \'algebra relacional}

   
    $$ \pi_{\text{\#J}, \text{Nombre}}(\text{Jugador} \Join (\text{Coleccionar}\,\sigma\,(\text{\#C}=2))) $$

    \begin{block}{¿En qu\'e orden se ejecutan las operaciones?}
        \onslide<2->{
        \begin{enumerate}
            \item $\text{Coleccionar} \,\sigma\, (\text{\#C = 2})$\hspace{4mm} \onslide<3->{$O(|\text{Coleccionar}|)$}
            \item $\text{Jugador} \Join (\text{Coleccionar} \,\sigma\,(\text{\#C} = 2))$\hspace{4mm} \onslide<4->{$O(|\text{Jugador}|^2)$}
            \item $\pi_{\text{\#J}, \text{Nombre}}(\text{Jugador} \Join (\text{Coleccionar}\,\sigma\,(\text{\#C}=2)))$ \hspace{4mm} \onslide<5->{$O  (|\text{Jugador}|)$}
        \end{enumerate}
        }

        \onslide<6>{

            $$
            O( |\text{Jugador}|^2 + |\text{Coleccionar}| +|\text{Jugador}| )
            $$
        }
    \end{block}
         
    
\end{frame}


\begin{frame}{El caracter imperativo del \'algebra relacional}
    \centering
    \Large No permite que el usuario se abstraiga de la optimizaci\'on

\end{frame}

\begin{frame}{Lenguajes de consulta declarativo}

    \begin{block}{C\'alculo relacional}
        \begin{itemize}
            \item C\'alculo relacional de tuplas
            \item C\'alculo relacional de dominios
        \end{itemize}
    \end{block}

\end{frame}


\begin{frame}{C\'alculo relacional de tuplas}
    \begin{block}{Definici\'on}
        Una consulta en el c\'alculo relacional de tuplas se expresa
        de la forma:

        $$
            \{t \,|\, P(t)\}
        $$
        donde: \begin{itemize}
            \item $t$ es una variable que representa una tupla
            \item $P$ es una f\'ormula bien formada compuesta de \'atomos
        \end{itemize}
    \end{block}
\end{frame}






\begin{frame}{C\'alculo relacional de tuplas}

    \begin{block}{Consultas}
        \begin{enumerate}
            \item Obtener todos los jugadores cuyo nivel es al menos 10.
            \onslide<2->{$$\{t \,|\, t \in \text{Jugador} \land t[\text{Nivel}] \geq 10\}$$} 
            \item<3-> Obtener el nombre de todos los jugadores cuyo nivel es al menos 10.
            \onslide<4->{$$\{t \,|\, \exists\, j \in \text{Jugador}(t[\text{\#J}] = j[\text{\#J}] \land t[\text{Nombre}] = j[\text{Nombre}] \land j[\text{Nivel}] \geq 10)\}$$}
        \end{enumerate}
    \end{block}

\end{frame}


\begin{frame}{C\'alculo relacional de tuplas}
 
    \begin{block}{Consultas}
        \begin{enumerate}
            \setcounter{enumi}{2}
            \item Obtener el nombre de todos los jugadores que tienen la carta con identificador 2.
            \onslide<2>{

                \begin{scriptsize}
                    $$ \{t | \exists j \in \text{Jugador} (t[\text{\#J}] = j[\text{\#J}] \land  t[\text{Nombre} ]= j[\text{Nombre}] \land \exists c \in \text{Coleccionar} ( j[\text{\#J}] = c[\text{\#J}] \land c[\text{\#C}] = 2 ))\}$$
                    
                \end{scriptsize}
            }
                
            
        \end{enumerate}

        
        
    \end{block}

\end{frame}

\begin{frame}{C\'alculo relacional de dominios}
    \begin{block}{Definici\'on}
        Una consulta en el c\'alculo relacional de dominios se
        expresa de la forma
        $$
            \{<x_1,x_2,...,x_n\> | P(x_1,x_2,...,x_n)\}
        $$
        donde: \begin{itemize}
            \item $x_1,x_2,...,x_n$ son variables de dominio
            \item $P$ es una f\'ormula bien formada compuesta de \'atomos
        \end{itemize}
    \end{block}
\end{frame}


\begin{frame}{C\'alculo relacional de dominios}

    \begin{block}{Consultas}
        Sean las variables $a, b, c ,d ,e$ correspondientes a \#J, Nombre, Nivel, Trofeos, TrofeosMax en la relaci\'on Jugador.

        \begin{enumerate}
            \item Obtener todos los jugadores cuyo nivel es al menos 10.
            \onslide<2->{$$\{<a,b,c,d,e> \,|\,  <a,b,c,d,e> \in \text{Jugador} \land c \geq 10\}$$} 
            \item<3-> Obtener el nombre de todos los jugadores cuyo nivel es al menos 10.
            \onslide<4->{$$\{<a,b> \,|\, \exists c,d,e(<a,b,c,d,e> \in \text{Jugador} \land c \geq 10)\}$$}
        \end{enumerate}
    \end{block}

\end{frame}

\begin{frame}{C\'alculo relacional de dominios}
    
    \begin{block}{Consultas}
        Sean las variables $a, b, c ,d ,e$ correspondientes a \#J, Nombre, Nivel, Trofeos, TrofeosMax en la relaci\'on Jugador.
        Sea la variable $f$ correspondiente a \#C en la relaci\'on Coleccionar.
        \begin{enumerate}
            \item Obtener el nombre de todos los jugadores que tienen la carta con identificador 2.
            \onslide<2->{

                \begin{small}
                    $$\{<a,b> \,|\, \exists c,d,e(<a,b,c,d,e> \in \text{Jugador} \land \exists f(<a,f> \in Coleccionar \land f=2))\}$$ 
                    
                \end{small}
            }
                
            
        \end{enumerate}

        
        
    \end{block}

\end{frame}


\begin{frame}{Equivalencia entre lenguajes}
    \begin{block}{El teorema de Codd}
        \begin{itemize}
            \item Para toda expresi\'on $E$ del \'algebra
            relacional existe una consulta $Q$ del c\'alculo relacional
            tal que $E \equiv  Q$
            \item Para toda consulta $Q$ del c\'alculo
            relacional existe una expresi\'on algebraica $E$ 
            tal que $Q \equiv  E$
        \end{itemize}

        El c\'alculo relacional es un lenguaje relacional completo
    \end{block}

    \onslide<2->{

        \begin{block}{Optimizaci\'on de consultas}
            Transformar una consulta $Q$ del c\'alculo relacional en una expresi\'on $E$ del \'algebra relacional.
        \end{block}
    }
\end{frame}


\begin{frame}
    \vspace{10mm}

    \centering
    \begin{Huge}
        
        \textcolor{blue3}{Turing completo vs Relacional completo}
    \end{Huge}

    \onslide<2>{

        \begin{columns}[T]
            \begin{column}{0.48\linewidth}
                \begin{itemize}
                    \item Mayor expresividad
                    \item No puede ser optimizado para el caso general de forma autom\'atica.
                \end{itemize}
            \end{column}
    
            \begin{column}{0.48\linewidth}
                \begin{itemize}
                    \item Menor expresividad
                    \item Puede ser optimizado para el caso general de forma autom\'atica.
                \end{itemize}
            \end{column}
            
        \end{columns}
    }
\end{frame}


