\begin{frame}{Modelo relacional}
    \centering
    \Huge \textcolor{blue3}{Restricciones de integridad}
\end{frame}


\begin{frame}{¿Qu\'e es un estado consistente de la base de datos?}
    \begin{block}{Estado de una base de datos}
        \begin{itemize}
            \item  Conjunto de instancias $\{r_1,r_2,...,r_n\}$ de las relaciones $R_1,R_2,...,R_n$ respectivamente
            que conforman la base de datos en un instante de tiempo espec\'ifico.
            \item<2>  Un estado es consistente si satisface cada una de las restricciones de integridad
            definidas sobre la base de datos.
        \end{itemize}
       
    \end{block}
\end{frame}


% \begin{frame}{Metarreglas}
%     \begin{block}{Integridad de las entidades}
%         El valor de una llave primaria debe ser no nulo del todo 
%     \end{block}

%     \vspace{5mm}
%     \centering
%     \resizebox{!}{2.6cm}{
%         \begin{tikzpicture}[node distance=5em]
%             \tikzstyle{every entity} = [minimum width=2cm, minimum height=1.2cm]
%             \node[entity, double distance=1.5pt] (jugador) {TEMPORADA}
%                 [sibling distance=3cm]
%                 child {node[attribute] [above right of=jugador] {\tiny PREMIO}}
%                 child {node[attribute] [above of=jugador] {\tiny \underline{A\~NO}}}
%                 child {node[attribute] [above left of=jugador] {\tiny \underline{\#L}}}
%                 ;
          
%             \node[entity] (clan) at (8,0) {EQUIPO}
%             [sibling distance=3cm]
%             child {node[attribute] [above of=clan] {\tiny NOMBRE}}
%             child {node[attribute] [above left of=clan] {\tiny PA\'IS}}
%             child {node[attribute] [above right of=clan] {\tiny \underline{\#E}}};
%             \node[relationship,aspect=2] (pertenecer) at (4,0) {PARTICIPAR}
%             edge(jugador) edge(clan);
%         \end{tikzpicture}
%     }
% \end{frame}


\begin{frame}{\only<-3>{Metarreglas}\only<4>{La minimalidad}}
    \begin{block}{Integridad de las entidades}
        Todos los atributos de una llave primaria deben ser no nulos
    \end{block}
    \vspace{5mm}

    \begin{tikzpicture}
        \node at (0,0) {
            \begin{minipage}{0.50\textwidth}
                \centering
                Relaci\'on R\\[2mm]
                \begin{tabular}{|ccc|}
                    \hline
                    \{(\underline{A}:$\mathbb{N}$), & (\underline{B}:$\mathbb{N}$), & (C:$\mathbb{N}$)\}\\
                    \hline
                    \hline
                    \{({\color<3->{red}A:1}), & (B:2022), & (C:1000)\}\\
                    \{({\color<3->{red}A:1}), & (B:2021), & (C:1000)\}\\
                    \{(A:2), & (B:2022), & (C:1200)\}\\
                    \hline
                \end{tabular}
            \end{minipage}    
        };
        
        \onslide<2->{
            \node at (7.6, 0.2) {¿Se puede insertar esta tupla?};
            \node at (7.6,-0.3) {\{({\color<3->{red}A:1}),(B:NULL),(C:1000)\}};
        }
    \end{tikzpicture}
    \vspace{3mm}

    \centering
    \onslide<4>{
    \large \textcolor{red}{La llave es minimal. Si su valor est\'a incompleto entonces no es \'unico}
    }
\end{frame}

\begin{frame}{\only<-2>{Metarreglas}\only<3>{La minimalidad... otra vez}}
    \begin{block}{Integridad referencial}
        \begin{itemize}
            \item  Todos los atributos de una llave for\'anea deben ser no nulos o todos deben ser nulos.
            \item El valor de una llave for\'anea tiene que ser un valor existente de la llave primaria en la relaci\'on a la que hace referencia.
        \end{itemize}
    \end{block}

   \pause

    \begin{tikzpicture}
        \node at (0,0) {
            \begin{minipage}{0.50\textwidth}
                \centering
                Relaci\'on R\\[2mm]
                \begin{tabular}{|ccc|}
                    \hline
                    \{(\underline{A}:$\mathbb{N}$), & (\underline{B}:$\mathbb{N}$), & (C:$\mathbb{N}$)\}\\
                    \hline
                    \hline
                    \{(A:1), & ({\color<4>{red}B:2022}), & (C:1000)\}\\
                    \{(A:1), & (B:2021), & (C:1000)\}\\
                    \{(A:2), & ({\color<4>{red}B:2022}), & (C:1200)\}\\
                    \hline
                \end{tabular}
            \end{minipage}    
        };


        \node at (6.5,0) {
            \begin{minipage}{0.50\textwidth}
                \centering
                Relaci\'on S\\{\tiny FK: (A,B) REFERENCES R (A,B)}\\[2mm]
                \begin{tabular}{|ccc|}
                    \hline
                    \{(\underline{D}:$\mathbb{N}$), & (A:$\mathbb{N}$), & (B:$\mathbb{N}$)\}\\
                    \hline
                    \hline
                    \{(D:2), & (A:\only<2>{2}\only<3->{\textbf<3>{NULL}}), & ({\color<4>{red}B:2022})\}\\
                    \hline
                \end{tabular}
            \end{minipage}    
        };
        
    \end{tikzpicture}
    \vspace{3mm}

    \only<4>{
    \centering
    \large ¿A cu\'al tupla de la relaci\'on R referencia la tupla de la relaci\'on S?
    }

    \only<5>{
        \centering
        \large \textcolor{red}{El valor de una llave for\'anea es un valor de la llave primaria de otra tabla\\Tambi\'en le afecta la minimalidad.}}

\end{frame}


\begin{frame}{Metarreglas}
  

    \begin{block}{Integridad de los dominios}
        \begin{itemize}
            \item Todos los valores de un atributo de una
            relaci\'on tienen que provenir del dominio pertinente.
        \end{itemize}
        
    \end{block}
    \vspace{3mm}

    \centering
    \onslide<2>{
        \Large \textcolor{red}{Trivial}
    }

    \note<2>{@NOTE por qu\'e trivial? Poner contraejemplo.}

\end{frame}



\begin{frame}{Modelo relacional}
    \vspace{5mm}
    \begin{overlayarea}{\linewidth}{\textheight}
        \begin{block}{Estructura de datos}
           Relaci\'on
        \end{block}

        \begin{block}{Restricciones de integridad}
            \begin{itemize}
                \item Metarreglas
                \item Dependencias funcionales (\textit{spoiler})
                \item Restricciones del negocio (\textit{spoiler})
            \end{itemize}
            
        \end{block}
    \end{overlayarea}

    \note{@NOTE restricciones del negocio: todo no se puede representar hasta este punto (hay q programar)}
\end{frame}

