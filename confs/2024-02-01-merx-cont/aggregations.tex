\begin{frame}{Estructurando interrelaciones}

    Una {\color<2>{orange}carta}, cuando un {\color<2>{orange}jugador} {\color<2>{blue}la agrega a su colecci\'on}, tiene un {\color<2>{attr}nivel} inicial, y este
    puede ir aumentando con el objetivo de mejorar sus atributos y tener un mejor desempeño en
    el campo de batalla.
\end{frame}

\begin{frame}{¿Puede una interrelaci\'on tener atributos?}
    \begin{block}{Agregaci\'on}
        Es un conjunto de entidades que se deriva de una interrelación entre
        otros conjuntos de entidades.
    \end{block}

    \vspace{5mm}

    \centering
    \begin{tikzpicture}
        \tikzstyle{every entity} = [minimum width=2.3cm, minimum height=1.2cm]
        \node[rectangle,draw,minimum width=12cm, minimum height=2.8cm] at (4.5,-0.4) {};
        \node[entity] (jugador) at (0,0) {JUGADOR};
        \node[entity] (carta) at (9,0) {CARTA};
        \node[relationship, aspect=2] (coleccionar) at (4.5,0) {COLECCIONAR} edge(jugador) edge(carta);
        \node at (7.5,0.2) {$1,\ast$};
        \node at (1.5,0.2) {$0,\ast$};
        \node at (8,-1.4) {CARTA COLECCIONADA};
    \end{tikzpicture}
    \vspace{5mm}

    \centering
    \onslide<2>{\textcolor{red}{\Large Una agregaci\'on contiene una y s\'olo una interrelaci\'on}}
\end{frame}

\begin{frame}{Atributo de una agregaci\'on}
    \centering
    \begin{tikzpicture}
        \tikzstyle{every entity} = [minimum width=2.3cm, minimum height=1.2cm]
        \node[rectangle,draw,minimum width=12cm, minimum height=2.8cm] (coleccionada) at (4.5,-0.4) {};
        \node[entity] (jugador) at (0,0) {JUGADOR};
        \node[entity] (carta) at (9,0) {CARTA};
        \node[relationship, aspect=2] (coleccionar) at (4.5,0) {COLECCIONAR} edge(jugador) edge(carta);
        \node at (7.5,0.2) {$1,\ast$};
        \node at (1.5,0.2) {$0,\ast$};
        \node at (8,-1.4) {CARTA COLECCIONADA};
        \node[attribute,dashed] (cartaid) at (4.5,2.3) { \underline{CARTA\_ID}};
        \node[attribute,dashed] (ci) at (7,2.3) {\underline{CI}};
        \node[attribute] (nivel) at (1.5,2.3) {NIVEL}; 
        \draw (coleccionada.north) -- (cartaid.south);
        \draw (7,1) -- (ci.south);
        \draw (1.5,1) -- (nivel.south);

    \end{tikzpicture}
    \vspace{5mm}

    \centering
    \textcolor{red}{\Large La agregaci\'on hereda la llave de la interrelaci\'on que contiene}
\end{frame}

\begin{frame}{Interrelacionando interrelaciones}
    \centering
    \begin{tikzpicture}
        \tikzstyle{every entity} = [minimum width=2.3cm, minimum height=1.2cm]
        \node[rectangle,draw,minimum width=12cm, minimum height=2.8cm] (coleccionada) at (4.5,-0.4) {};
        \node[entity] (jugador) at (0,0) {JUGADOR};
        \node[entity] (carta) at (9,0) {CARTA};
        \node[relationship, aspect=2] (coleccionar) at (4.5,0) {COLECCIONAR} edge(jugador) edge(carta);
        \node at (7.5,0.2) {$1,\ast$};
        \node at (1.5,0.2) {$0,\ast$};
        \node at (8,-1.4) {CARTA COLECCIONADA};
        % \node[attribute] (nivel) at (4.5,2.3) { NIVEL};
        % \draw[thick] (coleccionada.north) -- (nivel.south);

        \node[entity] (clan) at (8,-3.2) {CLAN};
        \node[relationship,aspect=2] (donar) at (4.5,-3.2) {DONAR} edge(clan);
        \draw[thick] (coleccionada.south) -- (donar.north);
        \node at (4.9, -2.1) {$0,\ast$};
        \node at (6.5, -3) {$0,1$};

        \node[attribute,dashed] (cartaid) at (4.5,2.3) { \underline{CARTA\_ID}};
        \node[attribute,dashed] (ci) at (7,2.3) {\underline{CI}};
        \node[attribute] (nivel) at (1.5,2.3) {NIVEL}; 
        \draw (coleccionada.north) -- (cartaid.south);
        \draw (7,1) -- (ci.south);
        \draw (1.5,1) -- (nivel.south);
    \end{tikzpicture}

   
\end{frame}
