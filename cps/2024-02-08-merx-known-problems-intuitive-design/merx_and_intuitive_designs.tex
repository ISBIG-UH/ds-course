\section{Dise\~nos conceptual e intuitivo}

Obtenga un dise\~no conceptual para cada uno de los escenarios siguientes. Obtenga la base de datos relacional correspondiente de acuerdo al dise\~no intuitivo en cada caso.

\subsection{Compa\~n\'ia de aviaci\'on}
Una compa\~n\'ia de vuelos, la cual realiza vuelos cortos entre pa\'ises de la Uni\'on Europea, desea dise\~nar una base de datos para mantener un registro de sus servicios.

Sobre los pasajeros se conoce su nombre, su direcci\'on y su tel\'efono. Sobre los vuelos se conoce el aeropuerto de salida, el aeropuerto de llegada y el avi\'on que realiza el vuelo. Los vuelos realizan salidas varias veces a la semana, de cada salida se registra su fecha y hora de salida, pudiendo los pasajeros reservar boletos para estas salidas. Sobre los modelos de aviones se tiene un registro de su
fabricante, el identificador del modelo, el n\'umero de asientos y motores. Sobre el personal se conoce el n\'umero del empleado, su nombre, su direcci\'on y su salario. Es importante se\~nalar que hay un tipo particular de empleado que es el piloto, de quien se tiene informaci\'on acerca de los modelos de avi\'on que puede pilotar.

\subsection{Empresa}
Se quiere confeccionar una base de datos sobre el personal de una empresa, cont\'andose con la informaci\'on siguiente:

La empresa posee un conjunto de departamentos, cada uno de los cuales tiene un conjunto de empleados, un conjunto de proyectos y un conjunto de oficinas. Cada departamento tiene un n\'umero que lo identifica, una funci\'on fundamental que desarrolla y un jefe \'unico quien, a su vez, se considera como empleado de la empresa. De cada empleado se conoce su n\'umero de empleado, los proyectos en los que se mantiene activo y los que no, el n\'umero de su oficina y su n\'umero de tel\'efono. Un empleado en un proyecto se considera un proyectista. Los proyectistas reciben evaluaciones peri\'odicamente, estas pueden ser EXCELENTE, BIEN, REGULAR y MAL. De cada tipo de evaluaci\'on (EXCELENTE, BIEN, REGULAR y MAL) s\'olo interesa almacenar la \'ultima fecha en la que el proyectista la recibi\'o. Adem\'as, para cada proyecto se tiene su n\'umero de identificaci\'on, su t\'itulo, su presupuesto y el tema en que se enmarca. Por cada oficina se tiene su n\'umero, su lugar de ubicaci\'on y el \'area que ocupa. Cada oficina tiene un conjunto de tel\'efonos que le corresponde.

\subsection{Sistema bancario}
Se desea almacenar la informaci\'on correspondiente a ciertas actividades que se desarrollan en un banco. Interesa la informaci\'on sobre los clientes, sus cuentas y otros servicios que brinda el banco, como la asignaci\'on de cr\'editos y el control de las inversiones. De los clientes se conoce su identificador, su nombre y su direcci\'on. Los clientes pueden ser personas o instituciones. De las personas se conoce su fecha de nacimiento y su sexo. De las instituciones se conoce su representante. De las cuentas se conoce el n\'umero que las identifica, su saldo y el inter\'es que acumula, que a su vez, depende del saldo. Existen dos tipos de cuentas: las cuentas corrientes y las cuentas de ahorro. Cualquier tipo de cliente puede tener o no varias cuentas, sin embargo, s\'olo las instituciones pueden tener cuentas corrientes. A su vez, las cuentas pueden estar asociadas a m\'ultiples clientes conoci\'endose en cada caso el monto de dinero depositado por cada cliente en dicha cuenta. El banco puede asignar cr\'editos s\'olo a los clientes de tipo persona. Por cada cr\'edito otorgado a un cliente se conoce la fecha de otorgamiento, su monto, el por ciento del monto inicial a pagar en cada mensualidad y la cantidad de mensualidades. La cantidad de mensualidades depende del por ciento del monto a pagar. Una instituci\'on con una cuenta corriente constituye un inversionista. Un inversionista puede participar en una o varias inversiones. De una inversi\'on se conoce su c\'odigo, su director y su por ciento de riesgo. En una inversi\'on pueden participar varios inversionistas, cada uno de los cuales puede aportar capitales diferentes.

\subsection{Competencias de f\'utbol}
Se desea almacenar la informaci\'on sobre los resultados de las competencias europeas de f\'utbol.  Una competencia se desarrolla en una temporada determinada y dentro de una misma competencia se realizan juegos entre los diferentes equipos de la liga. De las temporadas se conoce la fecha de su comienzo y la fecha de su terminaci\'on, las que conjuntamente determinan el c\'odigo que las identifica y viceversa. De las competencias se conoce su identificador, nombre y patrocinador. De los equipos se conoce su nombre, su director t\'ecnico y su color distintivo. De los jugadores se conoce su nombre, su alias (si lo tiene, y que es \'unico para cada nombre de jugador), su edad, su peso y su talla. Un jugador puede jugar en varios equipos en temporadas diferentes, pero s\'olo puede jugar en un mismo equipo en una temporada dada. Por supuesto, un jugador puede jugar en el mismo equipo en varias temporadas diferentes. Adem\'as, existe un conjunto de funciones o posiciones que pueden desempe\~nar u ocupar los jugadores, de las cuales se conoce su identificador y su nombre.  En general, un jugador puede ocupar varias funciones o posiciones, para cada una de las cuales se almacena su rendimiento en la misma. Es importante destacar que una posici\'on puede ocuparse por varios jugadores. Antes del inicio de cada partido, el director t\'ecnico del equipo debe seleccionar una alineaci\'on inicial. Una alineaci\'on es un conjunto de 11 jugadores donde cada uno de ellos tiene definida una posici\'on para dicho partido. Asimismo, se desea almacenar el resultado de los juegos celebrados entre dos equipos cualesquiera. De cada juego celebrado se desea almacenar el equipo ganador, el marcador final del encuentro, la fecha de realizaci\'on del juego y la competencia en la que ocurri\'o. En una misma fecha pueden realizarse distintos juegos entre los equipos, pero nunca m\'as de un juego entre los dos mismos equipos.

\subsection{Pr\'estamos de pel\'iculas}

La Videoteca Nacional de Cuba ha decidido, para mejorar su servicio, emplear una base de datos para almacenar la informaci\'on referente a las pel\'iculas que ofrece en alquiler. A continuaci\'on, se describe este proceso: Una pel\'icula se caracteriza por su t\'itulo, nacionalidad, productora y a\~no de realizaci\'on. Del personal que interviene en la realizaci\'on de las pel\'iculas se conoce su nombre, edad, nacionalidad y sexo. Existen dos roles muy espec\'ificos que pueden desempe\~nar las personas dentro de la realizaci\'on de una pel\'icula. Est\'an los directores, de los cuales se almacena su tipo (AUTOR, COMERCIAL, INDEPENDIENTE, DE G\'ENERO, DE ACTORES, DE FOTOGRAF\'IA), y los actores, de los cuales se conoce su m\'etodo actoral. Una pel\'icula puede tener m\'ultiples directores y actores dentro de su personal de realizaci\'on, destac\'andose que los actores pueden participar en una pel\'icula asumiendo un papel prot\'agonico o secundario. De cada pel\'icula se dispone de uno o varios ejemplares, diferenciados por un n\'umero de ejemplar y caracterizados por su estado de conservaci\'on.  Un ejemplar se puede encontrar alquilado a alg\'un asociado, de estos \'ultimos se sabe su n\'umero de identidad, nombre, direcci\'on y tel\'efono. Por cada alquiler se desea almacenar las fechas de comienzo y de devoluci\'on. La fecha de devoluci\'on del ejemplar siempre depende de la fecha de comienzo del alquiler. Un asociado tiene que ser avalado por otro asociado que responda por \'el en caso de alg\'un problema. Regularmente se realiza la presentaci\'on de una pel\'icula por uno de sus directores en alguna sala de video perteneciente al Proyecto 23 del ICAIC. De dichas salas se conoce su direcci\'on y capacidad de espectadores. A este tipo de actividad s\'olo puede asistir un selecto grupo de los asociados, conocidos como ``Cin\'efilos Estrella'', y de la misma se conoce la fecha de realizaci\'on.