\begin{frame}
    \frametitle{Ejercicios}
    \framesubtitle{6. ?`FN?}

    Considere el siguiente esquema relacional R(U, F), donde:

    U = $\{$A, B, C, D, E, G$\}$ y

    F = $\{$ AB $\rightarrow$ C, C $\rightarrow$ A, BC $\rightarrow$ D, D $\rightarrow$ EG, BE $\rightarrow$ C, CG $\rightarrow$ B, CE $\rightarrow$ G $\}$

    \begin{enumerate}
        \item[a)] Analice en qu\'e forma normal se encuentra R(U, F).
    \end{enumerate}

\end{frame}

\begin{frame}
    \frametitle{Ejercicios}
    \framesubtitle{7. Ahora con una tablita}

    Dado el siguiente esquema relacional R(U, F), donde:

    U = $\{$ CI, Nombre, NoCuenta, Banco, CodBanco $\}$ y

    F = $\{$ CI $\rightarrow$ Nombre, NoCuenta, CodBanco;
            CodBanco $\rightarrow$ Banco $\}$

    \begin{table}[!h]
        \small 
        \centering
        \begin{tabular}{|c|c|c|c|c|}
            \hline
            \textbf{CI} & \textbf{Nombre} & \textbf{NoCuenta} & \textbf{CodBanco} & \textbf{Banco}\\ \hline
            
            99031603817 & Luc\'ia & 0596 9143 9134 2891 & 634 & BM Carlos 3ro\\ \hline
            
            97011344519 & Alexa & 0585 3772 8749 1234 & 636 & BM Belascoain\\ \hline	
            
            93032683916 & Eduardo & 0512 3142 8246 1823 & 634 & BM Carlos 3ro\\ \hline	
            
            89032182713 & Amelia & 0533 3567 2243 8821 & 637 & BM CN Miramar\\ \hline	
        \end{tabular}
    \end{table}
        
    \begin{enumerate}
        \item[a)] Analice en qu\'e FN se encuentra la relaci\'on.
        
        \item[b)] \textquestiondown Se producen anomal\'ias en este ejemplo? Si la respuesta es afirmativa, ejemplifique cu\'ales.
        
        \item[c)] En funci\'on de lo analizado en el inciso anterior, proponga una descomposici\'on en la que no se produzcan tales anomal\'ias.
    \end{enumerate}

\end{frame}