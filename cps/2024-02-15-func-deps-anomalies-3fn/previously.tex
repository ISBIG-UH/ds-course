\begin{frame}
    \frametitle{Anteriormente... en Bases de Datos...}
    \framesubtitle{Dependencia Funcional}
    
    Dada una relaci\'on $R$ y los atributos $X$, $Y$ de $R$, se dice que $Y$ depende funcionalmente de $X$ si
    y s\'olo si el valor de $X$ en cada tupla de $R$ determina el valor
    de $Y$ en dicha tupla, o sea, si para todo par de tuplas $t_1, t_2$ se cumple que 
    
            $$
            t_1[X] = t_2[X] \implies t_1[Y] = t_2[Y] 
            $$

    Se representa como $R.X \to R.Y$ o simplemente

    \begin{Huge}
        
        $$
            X \to Y
        $$
    \end{Huge}

    \begin{block}{Notaci\'on}
        \begin{itemize}
            \item Atributo simple: $A,B,C,D,E$
            \item Atributo compuesto (conjunto de atributos simples): $W,X,Y,Z$
        \end{itemize}
    \end{block}
\end{frame}

\begin{frame}
    \frametitle{Anteriormente... en Bases de Datos...}
    \framesubtitle{Mejorando la definici\'on de llave candidata}

    Sea $K$ un conjunto de atributos $\{A_1, A_2, ..., A_n\}$ de una esquema
    relacional $R(U,F)$ es llave candidata candidata del esquema
    si cumple las siguientes propiedades:
    \begin{enumerate}
        \item \textbf{Unicidad}: $K^+_F = U$
        \item \textbf{Minimalidad}: Ning\'un subconjunto propio de $K$ tiene la propiedad de unicidad.
    \end{enumerate}

    

\end{frame}