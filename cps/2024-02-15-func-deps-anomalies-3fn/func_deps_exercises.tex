\begin{frame}
    \frametitle{Ejercicios}
    \framesubtitle<2->{1. Identificando llaves candidatas}
    
    \only<2->{
    Considere la relación $R(U, F)$, tal que 
    $$U = \{A,B,C,D,E\} \quad \textrm{ y } \quad F = \{D \rightarrow C, \ CE \rightarrow A, \ D \rightarrow A, \ AE \rightarrow D \}.$$

    ¿Cuáles de los
    siguientes conjuntos de atributos constituyen llaves candidatas?

    \begin{center}
        CE, BDE, BD, CDE, AD, BCE y A.
    \end{center}

    }

\end{frame}

\begin{frame}
    \frametitle{Ejercicios}
    \framesubtitle{2. Detractores de $E$}

    Considere la relación $R(U, F)$, con $U=\{A,B,C,D,E\}$, que satisface el conjunto de
    dependencias funcionales $F = \{ AB \rightarrow C, \ C \rightarrow D, \ BD \rightarrow E \}$. ¿Cuáles de los siguientes
    conjuntos de atributos no determinan funcionalmente a E?

    \begin{center}
        ABC, AB, BC, AD, ACD, BE y C.
    \end{center}

\end{frame}

\begin{frame}
    \frametitle{Ejercicios}
    \framesubtitle{3. Con $R$ hay que ser legal}

    Suponga que el conjunto universo $U$ de una relación $R$ es $U = \{A,B,C\}$. Actualmente, $R$ s\'olo contiene a la tupla
    $(0,0,0)$, y dicha relación siempre satisface las dependencias funcionales $A \rightarrow B$ y $B \rightarrow C$. ¿Cuáles de las siguientes tuplas pudieran ser insertadas en $R$?
    \begin{center}
        (0,1,0), (0,0,2), (1,1,0), (1,0,2), (0,1,2), (1,2,0) y (1,0,1).
    \end{center}

\end{frame}

\begin{frame}
    \frametitle{Ejercicios}
    \framesubtitle{4. Aprovechando los vac\'ios legales}

    Se conoce de la existencia de una relación $R(U, F)$, donde 

    $$U = \{A,B,C,D,E\} \quad \textrm{ y } \quad F = \{ A \rightarrow B, \ B \rightarrow C, \ DE \rightarrow A \},$$
    pero hemos perdido los valores de ciertas tuplas. ?`Podr\'ias ayudarnos a completarlas? Ten en cuenta que los atributos pueden tomar valores entre 0 y 3.

    \begin{center}
        (0,0,0,1,1), \ (2,1,0,0,3), \ (1,2,0,1,0), \ (1,2, NULL,0,1), \ (3,3,2,0, NULL),\\ (2, NULL, NULL,3,0) 
    \end{center}

\end{frame}

\begin{frame}
    \frametitle{Ejercicios}
    \framesubtitle{5. Aprovechando los vac\'ios legales (la secuela)}

    ¿De verdad entendieron lo que son las dependencias funcionales? Se conoce de la existencia de una relación R(A,B,C,D,E) que satisface las dependencias funcionales:

    $\{ CE \rightarrow A, AE \rightarrow D, A \rightarrow E, D \rightarrow CA \}$

    Pero somos muy despistados y nuevamente hemos perdido valores de ciertas tuplas. ?`Podr\'ias sugerirnos cómo completarlas? Los atributos pueden tomar valores entre 1 y 20.

    %$(2, 1, \gamma, 1, 3), (7, 15, 6, \Delta, 3), (2, 5, \omega, \Omega, \delta), (\alpha, 4, 9, 1, 3)$

    \begin{center}
        (2,1, NULL,1,3), (7,15,6, NULL,3), (2,5, NULL, NULL, NULL),\\
        (NULL,4,9,1,3)  
    \end{center}

\end{frame}